\begin{figure*}[ht]
\center
\tikz\node[draw=black!40!lightblue,inner sep=1pt,line width=0.3mm,rounded corners=0.1cm]{
\begin{tabular}{p{.95\textwidth}}
\textbf{Wikipedia Page}: \underline{Fresno,\_California} \textbf{Question ID}: 5725fe63ec44d21400f3d7de \\
\textbf{Question}: In what year was the Interstate Highway System created? \\
\textbf{Official Answer}: 1950s \textbf{|} in the 1950s \\
\textbf{Context}: Fresno is the largest U.S. city not directly linked to an Interstate highway. When the Interstate Highway System was created in the 1950s, the decision was made to build what is now Interstate 5 on the west side of the Central Valley, and thus bypass many of the population centers in the region, instead of upgrading what is now State Route 99. Due to rapidly raising population and traffic in cities along SR 99, as well as the desirability of Federal funding, much discussion has been made to upgrade it to interstate standards and eventually incorporate it into the interstate system, most likely as Interstate 9. Major improvements to signage, lane width, median separation, vertical clearance, and other concerns are currently underway.\\
\textbf{Validity}: correct \textbf{Reason}:  \textbf{Rater}: me@pedro.ai\\
\textbf{Validity}: flawed \textbf{Reason}: answer\_partially\_correct \textbf{Rater}: joseph.d.barrow@gmail.com\\
\end{tabular}
};
\label{fig:ex-5725fe63ec44d21400f3d7de}
\end{figure*}

\begin{figure*}[ht]
\center
\tikz\node[draw=black!40!lightblue,inner sep=1pt,line width=0.3mm,rounded corners=0.1cm]{
\begin{tabular}{p{.95\textwidth}}
\textbf{Wikipedia Page}: \underline{Immune\_system} \textbf{Question ID}: 5729ffda1d046914007796b2 \\
\textbf{Question}: Vaccination exploits what feature of the human immune system in order to be successful? \\
\textbf{Official Answer}: natural specificity of the immune system \textbf{|} natural specificity \textbf{|} the natural specificity \\
\textbf{Context}: Long-term active memory is acquired following infection by activation of B and T cells. Active immunity can also be generated artificially, through vaccination. The principle behind vaccination (also called immunization) is to introduce an antigen from a pathogen in order to stimulate the immune system and develop specific immunity against that particular pathogen without causing disease associated with that organism. This deliberate induction of an immune response is successful because it exploits the natural specificity of the immune system, as well as its inducibility. With infectious disease remaining one of the leading causes of death in the human population, vaccination represents the most effective manipulation of the immune system mankind has developed.\\
\textbf{Validity}: correct \textbf{Reason}:  \textbf{Rater}: me@pedro.ai\\
\textbf{Validity}: flawed \textbf{Reason}: answer\_set\_incomplete \textbf{Rater}: joseph.d.barrow@gmail.com\\
\end{tabular}
};
\label{fig:ex-5729ffda1d046914007796b2}
\end{figure*}

\clearpage

\begin{figure*}[ht]
\center
\tikz\node[draw=black!40!lightblue,inner sep=1pt,line width=0.3mm,rounded corners=0.1cm]{
\begin{tabular}{p{.95\textwidth}}
\textbf{Wikipedia Page}: \underline{Amazon\_rainforest} \textbf{Question ID}: 5725c071271a42140099d12b \\
\textbf{Question}: Where did it join in the direction of its flow? \\
\textbf{Official Answer}: joining the easterly flow toward the Atlantic. \textbf{|} the easterly flow \textbf{|} easterly \\
\textbf{Context}: During the mid-Eocene, it is believed that the drainage basin of the Amazon was split along the middle of the continent by the Purus Arch. Water on the eastern side flowed toward the Atlantic, while to the west water flowed toward the Pacific across the Amazonas Basin. As the Andes Mountains rose, however, a large basin was created that enclosed a lake; now known as the Solimoes Basin. Within the last 5-10 million years, this accumulating water broke through the Purus Arch, joining the easterly flow toward the Atlantic.\\
\textbf{Validity}: correct \textbf{Reason}: ambiguous \textbf{Rater}: me@pedro.ai\\
\textbf{Validity}: flawed \textbf{Reason}: bad\_question \textbf{Rater}: joseph.d.barrow@gmail.com\\
\end{tabular}
};
\label{fig:ex-5725c071271a42140099d12b}
\end{figure*}

\clearpage

\begin{figure*}[ht]
\center
\tikz\node[draw=black!40!lightblue,inner sep=1pt,line width=0.3mm,rounded corners=0.1cm]{
\begin{tabular}{p{.95\textwidth}}
\textbf{Wikipedia Page}: \underline{Pharmacy} \textbf{Question ID}: 5726e3c4dd62a815002e9407 \\
\textbf{Question}: What do clinical pharmacists often participate in? \\
\textbf{Official Answer}: patient care rounds drug product selection \textbf{|} interdisciplinary approach \textbf{|} patient care rounds drug product selection \\
\textbf{Context}: Pharmacists provide direct patient care services that optimizes the use of medication and promotes health, wellness, and disease prevention. Clinical pharmacists care for patients in all health care settings, but the clinical pharmacy movement initially began inside hospitals and clinics. Clinical pharmacists often collaborate with physicians and other healthcare professionals to improve pharmaceutical care. Clinical pharmacists are now an integral part of the interdisciplinary approach to patient care. They often participate in patient care rounds drug product selection.\\
\textbf{Validity}: wrong \textbf{Reason}: answer\_set\_incomplete \textbf{Rater}: me@pedro.ai\\
\textbf{Validity}: flawed \textbf{Reason}: one\_answer\_wrong \textbf{Rater}: joseph.d.barrow@gmail.com\\
\end{tabular}
};
\label{fig:ex-5726e3c4dd62a815002e9407}
\end{figure*}

\clearpage

\begin{figure*}[ht]
\center
\tikz\node[draw=black!40!lightblue,inner sep=1pt,line width=0.3mm,rounded corners=0.1cm]{
\begin{tabular}{p{.95\textwidth}}
\textbf{Wikipedia Page}: \underline{French\_and\_Indian\_War} \textbf{Question ID}: 5733d13e4776f419006612c5 \\
\textbf{Question}: How successful was initial effort by Braddock? \\
\textbf{Official Answer}: disaster; he was defeated in the Battle of the Monongahela \textbf{|} disaster \textbf{|} was a disaster \textbf{|} he was defeated \textbf{|} None succeeded \\
\textbf{Context}: In 1755, six colonial governors in North America met with General Edward Braddock, the newly arrived British Army commander, and planned a four-way attack on the French. None succeeded and the main effort by Braddock was a disaster; he was defeated in the Battle of the Monongahela on July 9, 1755 and died a few days later. British operations in 1755, 1756 and 1757 in the frontier areas of Pennsylvania and New York all failed, due to a combination of poor management, internal divisions, and effective Canadian scouts, French regular forces, and Indian warrior allies. In 1755, the British captured Fort Beausejour on the border separating Nova Scotia from Acadia; soon afterward they ordered the expulsion of the Acadians. Orders for the deportation were given by William Shirley, Commander-in-Chief, North America, without direction from Great Britain. The Acadians, both those captured in arms and those who had sworn the loyalty oath to His Britannic Majesty, were expelled. Native Americans were likewise driven off their land to make way for settlers from New England.\\
\textbf{Validity}: wrong \textbf{Reason}: answer\_set\_incomplete \textbf{Rater}: me@pedro.ai\\
\textbf{Validity}: correct \textbf{Reason}:  \textbf{Rater}: joseph.d.barrow@gmail.com\\
\end{tabular}
};
\label{fig:ex-5733d13e4776f419006612c5}
\end{figure*}

\clearpage

\begin{figure*}[ht]
\center
\tikz\node[draw=black!40!lightblue,inner sep=1pt,line width=0.3mm,rounded corners=0.1cm]{
\begin{tabular}{p{.95\textwidth}}
\textbf{Wikipedia Page}: \underline{Rhine} \textbf{Question ID}: 572ffb02b2c2fd14005686b8 \\
\textbf{Question}: What elements from the rift system in the Alpine orogeny in Southwest Germany? \\
\textbf{Official Answer}: Upper Rhine Graben \textbf{|} Upper Rhine Graben \textbf{|} Upper Rhine Graben \\
\textbf{Context}: From the Eocene onwards, the ongoing Alpine orogeny caused a N-S rift system to develop in this zone. The main elements of this rift are the Upper Rhine Graben, in southwest Germany and eastern France and the Lower Rhine Embayment, in northwest Germany and the southeastern Netherlands. By the time of the Miocene, a river system had developed in the Upper Rhine Graben, that continued northward and is considered the first Rhine river. At that time, it did not yet carry discharge from the Alps; instead, the watersheds of the Rhone and Danube drained the northern flanks of the Alps.\\
\textbf{Validity}: correct \textbf{Reason}: incomplete\_answer \textbf{Rater}: me@pedro.ai\\
\textbf{Validity}: flawed \textbf{Reason}: bad\_question \textbf{Rater}: joseph.d.barrow@gmail.com\\
\end{tabular}
};
\label{fig:ex-572ffb02b2c2fd14005686b8}
\end{figure*}

\clearpage

\begin{figure*}[ht]
\center
\tikz\node[draw=black!40!lightblue,inner sep=1pt,line width=0.3mm,rounded corners=0.1cm]{
\begin{tabular}{p{.95\textwidth}}
\textbf{Wikipedia Page}: \underline{Computational\_complexity\_theory} \textbf{Question ID}: 5ad567055b96ef001a10adeb \\
\textbf{Question}: What theory is the Cobham-Edward thesis? \\
\textbf{Official Answer}: Not Answerable \\
\textbf{Context}: The complexity class P is often seen as a mathematical abstraction modeling those computational tasks that admit an efficient algorithm. This hypothesis is called the Cobham-Edmonds thesis. The complexity class NP, on the other hand, contains many problems that people would like to solve efficiently, but for which no efficient algorithm is known, such as the Boolean satisfiability problem, the Hamiltonian path problem and the vertex cover problem. Since deterministic Turing machines are special non-deterministic Turing machines, it is easily observed that each problem in P is also member of the class NP.\\
\textbf{Validity}: wrong \textbf{Reason}: is\_answerable \textbf{Rater}: me@pedro.ai\\
\textbf{Validity}: flawed \textbf{Reason}: bad\_question is\_answerable \textbf{Rater}: joseph.d.barrow@gmail.com\\
\end{tabular}
};
\label{fig:ex-5ad567055b96ef001a10adeb}
\end{figure*}

\clearpage

\begin{figure*}[ht]
\center
\tikz\node[draw=black!40!lightblue,inner sep=1pt,line width=0.3mm,rounded corners=0.1cm]{
\begin{tabular}{p{.95\textwidth}}
\textbf{Wikipedia Page}: \underline{Construction} \textbf{Question ID}: 57273f27dd62a815002e9a0c \\
\textbf{Question}: What has a classification system for construction companies? \\
\textbf{Official Answer}: The Standard Industrial Classification and the newer North American Industry Classification System \textbf{|} Standard Industrial Classification \textbf{|} The Standard Industrial Classification and the newer North American Industry Classification System \\
\textbf{Context}: The Standard Industrial Classification and the newer North American Industry Classification System have a classification system for companies that perform or otherwise engage in construction. To recognize the differences of companies in this sector, it is divided into three subsectors: building construction, heavy and civil engineering construction, and specialty trade contractors. There are also categories for construction service firms (e.g., engineering, architecture) and construction managers (firms engaged in managing construction projects without assuming direct financial responsibility for completion of the construction project).\\
\textbf{Validity}: wrong \textbf{Reason}: answer\_set\_incomplete \textbf{Rater}: me@pedro.ai\\
\textbf{Validity}: correct \textbf{Reason}:  \textbf{Rater}: joseph.d.barrow@gmail.com\\
\end{tabular}
};
\label{fig:ex-57273f27dd62a815002e9a0c}
\end{figure*}

\clearpage

\begin{figure*}[ht]
\center
\tikz\node[draw=black!40!lightblue,inner sep=1pt,line width=0.3mm,rounded corners=0.1cm]{
\begin{tabular}{p{.95\textwidth}}
\textbf{Wikipedia Page}: \underline{Amazon\_rainforest} \textbf{Question ID}: 5726722bdd62a815002e8529 \\
\textbf{Question}: How many tree species are in the rainforest? \\
\textbf{Official Answer}: 1,100 \textbf{|} more than 1,100 \textbf{|} more than 1,100 \textbf{|} 1,100 \\
\textbf{Context}: The biodiversity of plant species is the highest on Earth with one 2001 study finding a quarter square kilometer (62 acres) of Ecuadorian rainforest supports more than 1,100 tree species. A study in 1999 found one square kilometer (247 acres) of Amazon rainforest can contain about 90,790 tonnes of living plants. The average plant biomass is estimated at 356 +- 47 tonnes per hectare. To date, an estimated 438,000 species of plants of economic and social interest have been registered in the region with many more remaining to be discovered or catalogued. The total number of tree species in the region is estimated at 16,000.\\
\textbf{Validity}: flawed \textbf{Reason}: missing\_answer ambiguous \textbf{Rater}: me@pedro.ai\\
\textbf{Validity}: correct \textbf{Reason}: one\_answer\_wrong \textbf{Rater}: joseph.d.barrow@gmail.com\\
\end{tabular}
};
\label{fig:ex-5726722bdd62a815002e8529}
\end{figure*}

\clearpage

\begin{figure*}[ht]
\center
\tikz\node[draw=black!40!lightblue,inner sep=1pt,line width=0.3mm,rounded corners=0.1cm]{
\begin{tabular}{p{.95\textwidth}}
\textbf{Wikipedia Page}: \underline{Civil\_disobedience} \textbf{Question ID}: 5728e5224b864d1900165033 \\
\textbf{Question}: What is an example of illegal disobedience? \\
\textbf{Official Answer}: trespassing at a nuclear-missile installation \textbf{|} symbolic illegal protests \textbf{|} trespassing at a nuclear-missile installation \textbf{|} the proprietors of illegal medical cannabis dispensaries \textbf{|} trespassing at a nuclear-missile installation \\
\textbf{Context}: Civil disobedients have chosen a variety of different illegal acts. Bedau writes, "There is a whole class of acts, undertaken in the name of civil disobedience, which, even if they were widely practiced, would in themselves constitute hardly more than a nuisance (e.g. trespassing at a nuclear-missile installation)...Such acts are often just a harassment and, at least to the bystander, somewhat inane...The remoteness of the connection between the disobedient act and the objectionable law lays such acts open to the charge of ineffectiveness and absurdity." Bedau also notes, though, that the very harmlessness of such entirely symbolic illegal protests toward public policy goals may serve a propaganda purpose. Some civil disobedients, such as the proprietors of illegal medical cannabis dispensaries and Voice in the Wilderness, which brought medicine to Iraq without the permission of the U.S. Government, directly achieve a desired social goal (such as the provision of medication to the sick) while openly breaking the law. Julia Butterfly Hill lived in Luna, a 180-foot (55 m)-tall, 600-year-old California Redwood tree for 738 days, successfully preventing it from being cut down.\\
\textbf{Validity}: flawed \textbf{Reason}: answer\_partially\_correct \textbf{Rater}: me@pedro.ai\\
\textbf{Validity}: correct \textbf{Reason}:  \textbf{Rater}: joseph.d.barrow@gmail.com\\
\end{tabular}
};
\label{fig:ex-5728e5224b864d1900165033}
\end{figure*}

\clearpage

\begin{figure*}[ht]
\center
\tikz\node[draw=black!40!lightblue,inner sep=1pt,line width=0.3mm,rounded corners=0.1cm]{
\begin{tabular}{p{.95\textwidth}}
\textbf{Wikipedia Page}: \underline{Sky\_(United\_Kingdom)} \textbf{Question ID}: 5a2c30dabfd06b001a5aea2c \\
\textbf{Question}: Who reported that 17,000 customers received the service due to failed deliveries? \\
\textbf{Official Answer}: Not Answerable \\
\textbf{Context}: BSkyB launched its HDTV service, Sky+ HD, on 22 May 2006. Prior to its launch, BSkyB claimed that 40,000 people had registered to receive the HD service. In the week before the launch, rumours started to surface that BSkyB was having supply issues with its set top box (STB) from manufacturer Thomson. On Thursday 18 May 2006, and continuing through the weekend before launch, people were reporting that BSkyB had either cancelled or rescheduled its installation. Finally, the BBC reported that 17,000 customers had yet to receive the service due to failed deliveries. On 31 March 2012, Sky announced the total number of homes with Sky+HD was 4,222,000.\\
\textbf{Validity}: correct \textbf{Reason}:  \textbf{Rater}: me@pedro.ai\\
\textbf{Validity}: wrong \textbf{Reason}: is\_answerable \textbf{Rater}: joseph.d.barrow@gmail.com\\
\end{tabular}
};
\label{fig:ex-5a2c30dabfd06b001a5aea2c}
\end{figure*}

\clearpage

\begin{figure*}[ht]
\center
\tikz\node[draw=black!40!lightblue,inner sep=1pt,line width=0.3mm,rounded corners=0.1cm]{
\begin{tabular}{p{.95\textwidth}}
\textbf{Wikipedia Page}: \underline{Force} \textbf{Question ID}: 57376df3c3c5551400e51ed7 \\
\textbf{Question}: What can keep an object from moving when it is being pushed on a surface? \\
\textbf{Official Answer}: static friction \textbf{|} static friction \textbf{|} friction \textbf{|} static friction \textbf{|} applied force \\
\textbf{Context}: Pushing against an object on a frictional surface can result in a situation where the object does not move because the applied force is opposed by static friction, generated between the object and the table surface. For a situation with no movement, the static friction force exactly balances the applied force resulting in no acceleration. The static friction increases or decreases in response to the applied force up to an upper limit determined by the characteristics of the contact between the surface and the object.\\
\textbf{Validity}: correct \textbf{Reason}: one\_answer\_wrong \textbf{Rater}: me@pedro.ai\\
\textbf{Validity}: flawed \textbf{Reason}: one\_answer\_wrong \textbf{Rater}: joseph.d.barrow@gmail.com\\
\end{tabular}
};
\label{fig:ex-57376df3c3c5551400e51ed7}
\end{figure*}

\clearpage

\begin{figure*}[ht]
\center
\tikz\node[draw=black!40!lightblue,inner sep=1pt,line width=0.3mm,rounded corners=0.1cm]{
\begin{tabular}{p{.95\textwidth}}
\textbf{Wikipedia Page}: \underline{Civil\_disobedience} \textbf{Question ID}: 5728f50baf94a219006a9e55 \\
\textbf{Question}: What way do some people perform civil disobedience in a constructive way? \\
\textbf{Official Answer}: defiant speech \textbf{|} defiant speech \textbf{|} allocution \textbf{|} defiant speech \textbf{|} defiant speech \textbf{|} defiant speech \\
\textbf{Context}: Some civil disobedience defendants choose to make a defiant speech, or a speech explaining their actions, in allocution. In U.S. v. Burgos-Andujar, a defendant who was involved in a movement to stop military exercises by trespassing on U.S. Navy property argued to the court in allocution that "the ones who are violating the greater law are the members of the Navy". As a result, the judge increased her sentence from 40 to 60 days. This action was upheld because, according to the U.S. Court of Appeals for the First Circuit, her statement suggested a lack of remorse, an attempt to avoid responsibility for her actions, and even a likelihood of repeating her illegal actions. Some of the other allocution speeches given by the protesters complained about mistreatment from government officials.\\
\textbf{Validity}: flawed \textbf{Reason}: incomplete\_answer \textbf{Rater}: me@pedro.ai\\
\textbf{Validity}: correct \textbf{Reason}:  \textbf{Rater}: joseph.d.barrow@gmail.com\\
\end{tabular}
};
\label{fig:ex-5728f50baf94a219006a9e55}
\end{figure*}

\clearpage

\begin{figure*}[ht]
\center
\tikz\node[draw=black!40!lightblue,inner sep=1pt,line width=0.3mm,rounded corners=0.1cm]{
\begin{tabular}{p{.95\textwidth}}
\textbf{Wikipedia Page}: \underline{European\_Union\_law} \textbf{Question ID}: 5726a7ecf1498d1400e8e656 \\
\textbf{Question}: Which articles state that the member states' rights to deliver public services may not be obstructed? \\
\textbf{Official Answer}: Articles 106 and 107 \textbf{|} Articles 106 and 107 \textbf{|} Articles 106 and 107 \\
\textbf{Context}: Today, the Treaty of Lisbon prohibits anti-competitive agreements in Article 101(1), including price fixing. According to Article 101(2) any such agreements are automatically void. Article 101(3) establishes exemptions, if the collusion is for distributional or technological innovation, gives consumers a "fair share" of the benefit and does not include unreasonable restraints that risk eliminating competition anywhere (or compliant with the general principle of European Union law of proportionality). Article 102 prohibits the abuse of dominant position, such as price discrimination and exclusive dealing. Article 102 allows the European Council to regulations to govern mergers between firms (the current regulation is the Regulation 139/2004/EC). The general test is whether a concentration (i.e. merger or acquisition) with a community dimension (i.e. affects a number of EU member states) might significantly impede effective competition. Articles 106 and 107 provide that member state's right to deliver public services may not be obstructed, but that otherwise public enterprises must adhere to the same competition principles as companies. Article 107 lays down a general rule that the state may not aid or subsidise private parties in distortion of free competition and provides exemptions for charities, regional development objectives and in the event of a natural disaster.\\
\textbf{Validity}: wrong \textbf{Reason}: answer\_set\_incomplete \textbf{Rater}: me@pedro.ai\\
\textbf{Validity}: correct \textbf{Reason}:  \textbf{Rater}: joseph.d.barrow@gmail.com\\
\end{tabular}
};
\label{fig:ex-5726a7ecf1498d1400e8e656}
\end{figure*}

\clearpage

\begin{figure*}[ht]
\center
\tikz\node[draw=black!40!lightblue,inner sep=1pt,line width=0.3mm,rounded corners=0.1cm]{
\begin{tabular}{p{.95\textwidth}}
\textbf{Wikipedia Page}: \underline{Ctenophora} \textbf{Question ID}: 5726431d271a42140099d7f8 \\
\textbf{Question}: What event was blamed on the introduction of mnemiopsis into The Black Sea? \\
\textbf{Official Answer}: causing fish stocks to collapse \textbf{|} causing fish stocks to collapse \textbf{|} causing fish stocks to collapse \\
\textbf{Context}: Ctenophores may be abundant during the summer months in some coastal locations, but in other places they are uncommon and difficult to find. In bays where they occur in very high numbers, predation by ctenophores may control the populations of small zooplanktonic organisms such as copepods, which might otherwise wipe out the phytoplankton (planktonic plants), which are a vital part of marine food chains. One ctenophore, Mnemiopsis, has accidentally been introduced into the Black Sea, where it is blamed for causing fish stocks to collapse by eating both fish larvae and organisms that would otherwise have fed the fish. The situation was aggravated by other factors, such as over-fishing and long-term environmental changes that promoted the growth of the Mnemiopsis population. The later accidental introduction of Beroe helped to mitigate the problem, as Beroe preys on other ctenophores.\\
\textbf{Validity}: wrong \textbf{Reason}: no\_answer \textbf{Rater}: me@pedro.ai\\
\textbf{Validity}: correct \textbf{Reason}:  \textbf{Rater}: joseph.d.barrow@gmail.com\\
\end{tabular}
};
\label{fig:ex-5726431d271a42140099d7f8}
\end{figure*}

\clearpage

\begin{figure*}[ht]
\center
\tikz\node[draw=black!40!lightblue,inner sep=1pt,line width=0.3mm,rounded corners=0.1cm]{
\begin{tabular}{p{.95\textwidth}}
\textbf{Wikipedia Page}: \underline{Packet\_switching} \textbf{Question ID}: 5a551230134fea001a0e18d0 \\
\textbf{Question}: Late published versions were utilized by who? \\
\textbf{Official Answer}: Not Answerable \\
\textbf{Context}: DECnet is a suite of network protocols created by Digital Equipment Corporation, originally released in 1975 in order to connect two PDP-11 minicomputers. It evolved into one of the first peer-to-peer network architectures, thus transforming DEC into a networking powerhouse in the 1980s. Initially built with three layers, it later (1982) evolved into a seven-layer OSI-compliant networking protocol. The DECnet protocols were designed entirely by Digital Equipment Corporation. However, DECnet Phase II (and later) were open standards with published specifications, and several implementations were developed outside DEC, including one for Linux.\\
\textbf{Validity}: wrong \textbf{Reason}: is\_answerable \textbf{Rater}: me@pedro.ai\\
\textbf{Validity}: correct \textbf{Reason}:  \textbf{Rater}: joseph.d.barrow@gmail.com\\
\end{tabular}
};
\label{fig:ex-5a551230134fea001a0e18d0}
\end{figure*}

\clearpage

\begin{figure*}[ht]
\center
\tikz\node[draw=black!40!lightblue,inner sep=1pt,line width=0.3mm,rounded corners=0.1cm]{
\begin{tabular}{p{.95\textwidth}}
\textbf{Wikipedia Page}: \underline{Imperialism} \textbf{Question ID}: 573083dc2461fd1900a9ce6f \\
\textbf{Question}: Colonialism as a policy is caused by financial and what other reasons? \\
\textbf{Official Answer}: ideological \textbf{|} ideological \textbf{|} ideological \textbf{|} commercial \\
\textbf{Context}: The term "imperialism" is often conflated with "colonialism", however many scholars have argued that each have their own distinct definition. Imperialism and colonialism have been used in order to describe one's superiority, domination and influence upon a person or group of people. Robert Young writes that while imperialism operates from the center, is a state policy and is developed for ideological as well as financial reasons, colonialism is simply the development for settlement or commercial intentions. Colonialism in modern usage also tends to imply a degree of geographic separation between the colony and the imperial power. Particularly, Edward Said distinguishes the difference between imperialism and colonialism by stating; "imperialism involved 'the practice, the theory and the attitudes of a dominating metropolitan center ruling a distant territory', while colonialism refers to the 'implanting of settlements on a distant territory.' Contiguous land empires such as the Russian or Ottoman are generally excluded from discussions of colonialism.:116 Thus it can be said that imperialism includes some form of colonialism, but colonialism itself does not automatically imply imperialism, as it lacks a political focus.[further explanation needed]\\
\textbf{Validity}: flawed \textbf{Reason}: bad\_answers bad\_question \textbf{Rater}: me@pedro.ai\\
\textbf{Validity}: wrong \textbf{Reason}: bad\_answers no\_answer \textbf{Rater}: joseph.d.barrow@gmail.com\\
\end{tabular}
};
\label{fig:ex-573083dc2461fd1900a9ce6f}
\end{figure*}

\clearpage

\begin{figure*}[ht]
\center
\tikz\node[draw=black!40!lightblue,inner sep=1pt,line width=0.3mm,rounded corners=0.1cm]{
\begin{tabular}{p{.95\textwidth}}
\textbf{Wikipedia Page}: \underline{European\_Union\_law} \textbf{Question ID}: 572695285951b619008f774c \\
\textbf{Question}: What can block a legislation? \\
\textbf{Official Answer}: legislation can be blocked by a majority in Parliament, a minority in the Council, and a majority in the Commission \textbf{|} unanimity \textbf{|} unanimity \textbf{|} a majority in Parliament \\
\textbf{Context}: To make new legislation, TFEU article 294 defines the "ordinary legislative procedure" that applies for most EU acts. The essence is there are three readings, starting with a Commission proposal, where the Parliament must vote by a majority of all MEPs (not just those present) to block or suggest changes, and the Council must vote by qualified majority to approve changes, but by unanimity to block Commission amendment. Where the different institutions cannot agree at any stage, a "Conciliation Committee" is convened, representing MEPs, ministers and the Commission to try and get agreement on a joint text: if this works, it will be sent back to the Parliament and Council to approve by absolute and qualified majority. This means, legislation can be blocked by a majority in Parliament, a minority in the Council, and a majority in the Commission: it is harder to change EU law than stay the same. A different procedure exists for budgets. For "enhanced cooperation" among a sub-set of at least member states, authorisation must be given by the Council. Member state governments should be informed by the Commission at the outset before any proposals start the legislative procedure. The EU as a whole can only act within its power set out in the Treaties. TEU articles 4 and 5 state that powers remain with the member states unless they have been conferred, although there is a debate about the Kompetenz-Kompetenz question: who ultimately has the "competence" to define the EU's "competence". Many member state courts believe they decide, other member state Parliaments believe they decide, while within the EU, the Court of Justice believes it has the final say.\\
\textbf{Validity}: correct \textbf{Reason}:  \textbf{Rater}: me@pedro.ai\\
\textbf{Validity}: wrong \textbf{Reason}: answer\_partially\_correct \textbf{Rater}: joseph.d.barrow@gmail.com\\
\end{tabular}
};
\label{fig:ex-572695285951b619008f774c}
\end{figure*}

\clearpage

\begin{figure*}[ht]
\center
\tikz\node[draw=black!40!lightblue,inner sep=1pt,line width=0.3mm,rounded corners=0.1cm]{
\begin{tabular}{p{.95\textwidth}}
\textbf{Wikipedia Page}: \underline{Southern\_California} \textbf{Question ID}: 5705f09e75f01819005e77a4 \\
\textbf{Question}: Other than land laws, what else were the Californios dissatisfied with? \\
\textbf{Official Answer}: inequitable taxes \textbf{|} inequitable taxes \textbf{|} inequitable taxes \\
\textbf{Context}: Subsequently, Californios (dissatisfied with inequitable taxes and land laws) and pro-slavery southerners in the lightly populated "Cow Counties" of southern California attempted three times in the 1850s to achieve a separate statehood or territorial status separate from Northern California. The last attempt, the Pico Act of 1859, was passed by the California State Legislature and signed by the State governor John B. Weller. It was approved overwhelmingly by nearly 75\% of voters in the proposed Territory of Colorado. This territory was to include all the counties up to the then much larger Tulare County (that included what is now Kings, most of Kern, and part of Inyo counties) and San Luis Obispo County. The proposal was sent to Washington, D.C. with a strong advocate in Senator Milton Latham. However, the secession crisis following the election of Abraham Lincoln in 1860 led to the proposal never coming to a vote.\\
\textbf{Validity}: wrong \textbf{Reason}: answer\_set\_incomplete \textbf{Rater}: me@pedro.ai\\
\textbf{Validity}: correct \textbf{Reason}:  \textbf{Rater}: joseph.d.barrow@gmail.com\\
\end{tabular}
};
\label{fig:ex-5705f09e75f01819005e77a4}
\end{figure*}

\clearpage

\begin{figure*}[ht]
\center
\tikz\node[draw=black!40!lightblue,inner sep=1pt,line width=0.3mm,rounded corners=0.1cm]{
\begin{tabular}{p{.95\textwidth}}
\textbf{Wikipedia Page}: \underline{Geology} \textbf{Question ID}: 5a5909b13e1742001a15cf4c \\
\textbf{Question}: What kind of unit tends to change its volume? \\
\textbf{Official Answer}: Not Answerable \\
\textbf{Context}: When rock units are placed under horizontal compression, they shorten and become thicker. Because rock units, other than muds, do not significantly change in volume, this is accomplished in two primary ways: through faulting and folding. In the shallow crust, where brittle deformation can occur, thrust faults form, which cause deeper rock to move on top of shallower rock. Because deeper rock is often older, as noted by the principle of superposition, this can result in older rocks moving on top of younger ones. Movement along faults can result in folding, either because the faults are not planar or because rock layers are dragged along, forming drag folds as slip occurs along the fault. Deeper in the Earth, rocks behave plastically, and fold instead of faulting. These folds can either be those where the material in the center of the fold buckles upwards, creating "antiforms", or where it buckles downwards, creating "synforms". If the tops of the rock units within the folds remain pointing upwards, they are called anticlines and synclines, respectively. If some of the units in the fold are facing downward, the structure is called an overturned anticline or syncline, and if all of the rock units are overturned or the correct up-direction is unknown, they are simply called by the most general terms, antiforms and synforms.\\
\textbf{Validity}: correct \textbf{Reason}:  \textbf{Rater}: me@pedro.ai\\
\textbf{Validity}: flawed \textbf{Reason}: is\_answerable \textbf{Rater}: joseph.d.barrow@gmail.com\\
\end{tabular}
};
\label{fig:ex-5a5909b13e1742001a15cf4c}
\end{figure*}

\clearpage

\begin{figure*}[ht]
\center
\tikz\node[draw=black!40!lightblue,inner sep=1pt,line width=0.3mm,rounded corners=0.1cm]{
\begin{tabular}{p{.95\textwidth}}
\textbf{Wikipedia Page}: \underline{Economic\_inequality} \textbf{Question ID}: 572a1c943f37b319004786e3 \\
\textbf{Question}: Why did the demand for rentals decrease? \\
\textbf{Official Answer}: demand for higher quality housing increased \textbf{|} demand for higher quality housing increased \textbf{|} demand for higher quality housing increased \\
\textbf{Context}: A number of researchers (David Rodda, Jacob Vigdor, and Janna Matlack), argue that a shortage of affordable housing - at least in the US - is caused in part by income inequality. David Rodda noted that from 1984 and 1991, the number of quality rental units decreased as the demand for higher quality housing increased (Rhoda 1994:148). Through gentrification of older neighbourhoods, for example, in East New York, rental prices increased rapidly as landlords found new residents willing to pay higher market rate for housing and left lower income families without rental units. The ad valorem property tax policy combined with rising prices made it difficult or impossible for low income residents to keep pace.\\
\textbf{Validity}: flawed \textbf{Reason}: bad\_question \textbf{Rater}: me@pedro.ai\\
\textbf{Validity}: correct \textbf{Reason}:  \textbf{Rater}: joseph.d.barrow@gmail.com\\
\end{tabular}
};
\label{fig:ex-572a1c943f37b319004786e3}
\end{figure*}

\clearpage