\section{Negative Discriminability}

\begin{figure*}[ht]
    \center
    \tikz\node[draw=black!40!lightblue,inner sep=1pt,line width=0.3mm,rounded corners=0.1cm]{
        \begin{tabular}{p{.95\textwidth}}
            \textbf{\discability{}}: -9.63 \textbf{\diff{}}: -0.479 \textbf{Feasibility}: 0.614 \textbf{Mean Accuracy}: 0.472                                                                                                                                                                                                                                                                                                                                                                                                                                                                                                                                                                                                                                          \\
            \textbf{Validity}: flawed \textbf{Reason}: bad\_question                                                                                                                                                                                                                                                                                                                                                                                                                                                                                                                                                                                                                                                                                                   \\
            \textbf{Wikipedia Page}: \underline{Economic\_inequality} \textbf{Question ID}: 572a1c943f37b319004786e3                                                                                                                                                                                                                                                                                                                                                                                                                                                                                                                                                                                                                                                   \\
            \textbf{Question}: Why did the demand for rentals decrease?                                                                                                                                                                                                                                                                                                                                                                                                                                                                                                                                                                                                                                                                                                \\
            \textbf{Official Answer}: demand for higher quality housing increased \textbf{|} demand for higher quality housing increased \textbf{|} demand for higher quality housing increased                                                                                                                                                                                                                                                                                                                                                                                                                                                                                                                                                                        \\
            \textbf{Context}: A number of researchers (David Rodda, Jacob Vigdor, and Janna Matlack), argue that a shortage of affordable housing - at least in the US - is caused in part by income inequality. David Rodda noted that from 1984 and 1991, the number of quality rental units decreased as the demand for higher quality housing increased (Rhoda 1994:148). Through gentrification of older neighbourhoods, for example, in East New York, rental prices increased rapidly as landlords found new residents willing to pay higher market rate for housing and left lower income families without rental units. The ad valorem property tax policy combined with rising prices made it difficult or impossible for low income residents to keep pace. \\
        \end{tabular}
    };
    \label{fig:ex-572a1c943f37b319004786e3}
\end{figure*}

\begin{figure*}[ht]
    \center
    \tikz\node[draw=black!40!lightblue,inner sep=1pt,line width=0.3mm,rounded corners=0.1cm]{
        \begin{tabular}{p{.95\textwidth}}
            \textbf{\discability{}}: -9.09 \textbf{\diff{}}: -0.631 \textbf{Feasibility}: 0.677 \textbf{Mean Accuracy}: 0.472                                                                                                                                                                                                                                                                                                                                                                                                                                                                                                                                                                                                                                                                                                                                                                                                                                                                                                                                                                                                                                                                                                                                                                                                                    \\
            \textbf{Validity}: flawed \textbf{Reason}: bad\_question + bad\_answers                                                                                                                                                                                                                                                                                                                                                                                                                                                                                                                                                                                                                                                                                                                                                                                                                                                                                                                                                                                                                                                                                                                                                                                                                                                              \\
            \textbf{Wikipedia Page}: \underline{Imperialism} \textbf{Question ID}: 573083dc2461fd1900a9ce6f                                                                                                                                                                                                                                                                                                                                                                                                                                                                                                                                                                                                                                                                                                                                                                                                                                                                                                                                                                                                                                                                                                                                                                                                                                      \\
            \textbf{Question}: Colonialism as a policy is caused by financial and what other reasons?                                                                                                                                                                                                                                                                                                                                                                                                                                                                                                                                                                                                                                                                                                                                                                                                                                                                                                                                                                                                                                                                                                                                                                                                                                            \\
            \textbf{Official Answer}: ideological \textbf{|} ideological \textbf{|} ideological \textbf{|} commercial                                                                                                                                                                                                                                                                                                                                                                                                                                                                                                                                                                                                                                                                                                                                                                                                                                                                                                                                                                                                                                                                                                                                                                                                                            \\
            \textbf{Context}: The term "imperialism" is often conflated with "colonialism", however many scholars have argued that each have their own distinct definition. Imperialism and colonialism have been used in order to describe one's superiority, domination and influence upon a person or group of people. Robert Young writes that while imperialism operates from the center, is a state policy and is developed for ideological as well as financial reasons, colonialism is simply the development for settlement or commercial intentions. Colonialism in modern usage also tends to imply a degree of geographic separation between the colony and the imperial power. Particularly, Edward Said distinguishes the difference between imperialism and colonialism by stating; "imperialism involved 'the practice, the theory and the attitudes of a dominating metropolitan center ruling a distant territory', while colonialism refers to the 'implanting of settlements on a distant territory.' Contiguous land empires such as the Russian or Ottoman are generally excluded from discussions of colonialism.:116 Thus it can be said that imperialism includes some form of colonialism, but colonialism itself does not automatically imply imperialism, as it lacks a political focus.[further explanation needed] \\
        \end{tabular}
    };
    \label{fig:ex-573083dc2461fd1900a9ce6f}
\end{figure*}

\begin{figure*}[ht]
    \center
    \tikz\node[draw=black!40!lightblue,inner sep=1pt,line width=0.3mm,rounded corners=0.1cm]{
        \begin{tabular}{p{.95\textwidth}}
            \textbf{\discability{}}: -8.91 \textbf{\diff{}}: -0.334 \textbf{Feasibility}: 0.648 \textbf{Mean Accuracy}: 0.460                                                                                                                                                                                                                                                                                                                                                                                                                                                                                                                                                                                                                                                                                                                                                                                                                                                                                                                                                            \\
            \textbf{Validity}: wrong \textbf{Reason}: is\_answerable                                                                                                                                                                                                                                                                                                                                                                                                                                                                                                                                                                                                                                                                                                                                                                                                                                                                                                                                                                                                                     \\
            \textbf{Wikipedia Page}: \underline{University\_of\_Chicago} \textbf{Question ID}: 5acf782877cf76001a684eb4                                                                                                                                                                                                                                                                                                                                                                                                                                                                                                                                                                                                                                                                                                                                                                                                                                                                                                                                                                  \\
            \textbf{Question}: Under which field did the world's first synthetically produced nuclear reaction occur?                                                                                                                                                                                                                                                                                                                                                                                                                                                                                                                                                                                                                                                                                                                                                                                                                                                                                                                                                                    \\
            \textbf{Official Answer}: Not Answerable                                                                                                                                                                                                                                                                                                                                                                                                                                                                                                                                                                                                                                                                                                                                                                                                                                                                                                                                                                                                                                     \\
            \textbf{Context}: University of Chicago scholars have played a major role in the development of various academic disciplines, including: the Chicago school of economics, the Chicago school of sociology, the law and economics movement in legal analysis, the Chicago school of literary criticism, the Chicago school of religion, and the behavioralism school of political science. Chicago's physics department helped develop the world's first man-made, self-sustaining nuclear reaction beneath the university's Stagg Field. Chicago's research pursuits have been aided by unique affiliations with world-renowned institutions like the nearby Fermilab and Argonne National Laboratory, as well as the Marine Biological Laboratory. The university is also home to the University of Chicago Press, the largest university press in the United States. With an estimated completion date of 2020, the Barack Obama Presidential Center will be housed at the university and include both the Obama presidential library and offices of the Obama Foundation. \\
        \end{tabular}
    };
    \label{fig:ex-5acf782877cf76001a684eb4}
\end{figure*}

\begin{figure*}[ht]
    \center
    \tikz\node[draw=black!40!lightblue,inner sep=1pt,line width=0.3mm,rounded corners=0.1cm]{
        \begin{tabular}{p{.95\textwidth}}
            \textbf{\discability{}}: -8.51 \textbf{\diff{}}: -0.340 \textbf{Feasibility}: 0.561 \textbf{Mean Accuracy}: 0.447                                                                                                                                                                                                                                                                                                                                                                                                                                                                                                                                    \\
            \textbf{Validity}: wrong \textbf{Reason}: is\_answerable                                                                                                                                                                                                                                                                                                                                                                                                                                                                                                                                                                                             \\
            \textbf{Wikipedia Page}: \underline{Steam\_engine} \textbf{Question ID}: 5ad3cab2604f3c001a3ff0e3                                                                                                                                                                                                                                                                                                                                                                                                                                                                                                                                                    \\
            \textbf{Question}: Who invented a high-pressure power source around 1800?                                                                                                                                                                                                                                                                                                                                                                                                                                                                                                                                                                            \\
            \textbf{Official Answer}: Not Answerable                                                                                                                                                                                                                                                                                                                                                                                                                                                                                                                                                                                                             \\
            \textbf{Context}: Around 1800 Richard Trevithick and, separately, Oliver Evans in 1801 introduced engines using high-pressure steam; Trevithick obtained his high-pressure engine patent in 1802. These were much more powerful for a given cylinder size than previous engines and could be made small enough for transport applications. Thereafter, technological developments and improvements in manufacturing techniques (partly brought about by the adoption of the steam engine as a power source) resulted in the design of more efficient engines that could be smaller, faster, or more powerful, depending on the intended application. \\
        \end{tabular}
    };
    \label{fig:ex-5ad3cab2604f3c001a3ff0e3}
\end{figure*}

\begin{figure*}[ht]
    \center
    \tikz\node[draw=black!40!lightblue,inner sep=1pt,line width=0.3mm,rounded corners=0.1cm]{
        \begin{tabular}{p{.95\textwidth}}
            \textbf{\discability{}}: -8.35 \textbf{\diff{}}: -1.47 \textbf{Feasibility}: 0.560 \textbf{Mean Accuracy}: 0.112                                                                                                                                                                                                                                                                                                                                                                                                                                                                                                                                                                                                                                                                                                                                                                                                                                                                                                                                                                                                                                                                                                                                                                                                                                                                                                                                                                                                                                                                                                                                                                                                                                           \\
            \textbf{Validity}: correct \textbf{Reason}: NA                                                                                                                                                                                                                                                                                                                                                                                                                                                                                                                                                                                                                                                                                                                                                                                                                                                                                                                                                                                                                                                                                                                                                                                                                                                                                                                                                                                                                                                                                                                                                                                                                                                                                                             \\
            \textbf{Wikipedia Page}: \underline{European\_Union\_law} \textbf{Question ID}: 572695285951b619008f774c                                                                                                                                                                                                                                                                                                                                                                                                                                                                                                                                                                                                                                                                                                                                                                                                                                                                                                                                                                                                                                                                                                                                                                                                                                                                                                                                                                                                                                                                                                                                                                                                                                                   \\
            \textbf{Question}: What can block a legislation?                                                                                                                                                                                                                                                                                                                                                                                                                                                                                                                                                                                                                                                                                                                                                                                                                                                                                                                                                                                                                                                                                                                                                                                                                                                                                                                                                                                                                                                                                                                                                                                                                                                                                                           \\
            \textbf{Official Answer}: legislation can be blocked by a majority in Parliament, a minority in the Council, and a majority in the Commission \textbf{|} unanimity \textbf{|} unanimity \textbf{|} a majority in Parliament                                                                                                                                                                                                                                                                                                                                                                                                                                                                                                                                                                                                                                                                                                                                                                                                                                                                                                                                                                                                                                                                                                                                                                                                                                                                                                                                                                                                                                                                                                                                \\
            \textbf{Context}: To make new legislation, TFEU article 294 defines the "ordinary legislative procedure" that applies for most EU acts. The essence is there are three readings, starting with a Commission proposal, where the Parliament must vote by a majority of all MEPs (not just those present) to block or suggest changes, and the Council must vote by qualified majority to approve changes, but by unanimity to block Commission amendment. Where the different institutions cannot agree at any stage, a "Conciliation Committee" is convened, representing MEPs, ministers and the Commission to try and get agreement on a joint text: if this works, it will be sent back to the Parliament and Council to approve by absolute and qualified majority. This means, legislation can be blocked by a majority in Parliament, a minority in the Council, and a majority in the Commission: it is harder to change EU law than stay the same. A different procedure exists for budgets. For "enhanced cooperation" among a sub-set of at least member states, authorisation must be given by the Council. Member state governments should be informed by the Commission at the outset before any proposals start the legislative procedure. The EU as a whole can only act within its power set out in the Treaties. TEU articles 4 and 5 state that powers remain with the member states unless they have been conferred, although there is a debate about the Kompetenz-Kompetenz question: who ultimately has the "competence" to define the EU's "competence". Many member state courts believe they decide, other member state Parliaments believe they decide, while within the EU, the Court of Justice believes it has the final say. \\
        \end{tabular}
    };
    \label{fig:ex-572695285951b619008f774c}
\end{figure*}

\begin{figure*}[ht]
    \center
    \tikz\node[draw=black!40!lightblue,inner sep=1pt,line width=0.3mm,rounded corners=0.1cm]{
        \begin{tabular}{p{.95\textwidth}}
            \textbf{\discability{}}: -7.65 \textbf{\diff{}}: -1.02 \textbf{Feasibility}: 0.485 \textbf{Mean Accuracy}: 0.280                                                                                                                                                                                                                                                                                                                                                                                                                                                                                                                                                                      \\
            \textbf{Validity}: correct \textbf{Reason}: NA                                                                                                                                                                                                                                                                                                                                                                                                                                                                                                                                                                                                                                        \\
            \textbf{Wikipedia Page}: \underline{Force} \textbf{Question ID}: 5737a0acc3c5551400e51f49                                                                                                                                                                                                                                                                                                                                                                                                                                                                                                                                                                                             \\
            \textbf{Question}: In what kind of fluid are pressure differences caused by direction of forces over gradients?                                                                                                                                                                                                                                                                                                                                                                                                                                                                                                                                                                       \\
            \textbf{Official Answer}: extended \textbf{|} extended \textbf{|} extended                                                                                                                                                                                                                                                                                                                                                                                                                                                                                                                                                                                                            \\
            \textbf{Context}: Newton's laws and Newtonian mechanics in general were first developed to describe how forces affect idealized point particles rather than three-dimensional objects. However, in real life, matter has extended structure and forces that act on one part of an object might affect other parts of an object. For situations where lattice holding together the atoms in an object is able to flow, contract, expand, or otherwise change shape, the theories of continuum mechanics describe the way forces affect the material. For example, in extended fluids, differences in pressure result in forces being directed along the pressure gradients as follows: \\
        \end{tabular}
    };
    \label{fig:ex-5737a0acc3c5551400e51f49}
\end{figure*}

\begin{figure*}[ht]
    \center
    \tikz\node[draw=black!40!lightblue,inner sep=1pt,line width=0.3mm,rounded corners=0.1cm]{
        \begin{tabular}{p{.95\textwidth}}
            \textbf{\discability{}}: -7.45 \textbf{\diff{}}: -0.309 \textbf{Feasibility}: 0.624 \textbf{Mean Accuracy}: 0.472                                                                                                                                                                                                                                                                                                                                                                                                                                                                                                                                                      \\
            \textbf{Validity}: flawed \textbf{Reason}: ambiguous + missing\_answer                                                                                                                                                                                                                                                                                                                                                                                                                                                                                                                                                                                                 \\
            \textbf{Wikipedia Page}: \underline{Amazon\_rainforest} \textbf{Question ID}: 5726722bdd62a815002e8529                                                                                                                                                                                                                                                                                                                                                                                                                                                                                                                                                                 \\
            \textbf{Question}: How many tree species are in the rainforest?                                                                                                                                                                                                                                                                                                                                                                                                                                                                                                                                                                                                        \\
            \textbf{Official Answer}: 1,100 \textbf{|} more than 1,100 \textbf{|} more than 1,100 \textbf{|} 1,100                                                                                                                                                                                                                                                                                                                                                                                                                                                                                                                                                                 \\
            \textbf{Context}: The biodiversity of plant species is the highest on Earth with one 2001 study finding a quarter square kilometer (62 acres) of Ecuadorian rainforest supports more than 1,100 tree species. A study in 1999 found one square kilometer (247 acres) of Amazon rainforest can contain about 90,790 tonnes of living plants. The average plant biomass is estimated at 356 +- 47 tonnes per hectare. To date, an estimated 438,000 species of plants of economic and social interest have been registered in the region with many more remaining to be discovered or catalogued. The total number of tree species in the region is estimated at 16,000. \\
        \end{tabular}
    };
    \label{fig:ex-5726722bdd62a815002e8529}
\end{figure*}

\begin{figure*}[ht]
    \center
    \tikz\node[draw=black!40!lightblue,inner sep=1pt,line width=0.3mm,rounded corners=0.1cm]{
        \begin{tabular}{p{.95\textwidth}}
            \textbf{\discability{}}: -7.43 \textbf{\diff{}}: -0.368 \textbf{Feasibility}: 0.582 \textbf{Mean Accuracy}: 0.422                                                                                                                                                                                                                                                                                                                                                                                                                                                                                                                                                                                                                                                                                                                                                                                                                                 \\
            \textbf{Validity}: wrong \textbf{Reason}: not\_answerable                                                                                                                                                                                                                                                                                                                                                                                                                                                                                                                                                                                                                                                                                                                                                                                                                                                                                         \\
            \textbf{Wikipedia Page}: \underline{Ctenophora} \textbf{Question ID}: 5726431d271a42140099d7f8                                                                                                                                                                                                                                                                                                                                                                                                                                                                                                                                                                                                                                                                                                                                                                                                                                                    \\
            \textbf{Question}: What event was blamed on the introduction of mnemiopsis into The Black Sea?                                                                                                                                                                                                                                                                                                                                                                                                                                                                                                                                                                                                                                                                                                                                                                                                                                                    \\
            \textbf{Official Answer}: causing fish stocks to collapse \textbf{|} causing fish stocks to collapse \textbf{|} causing fish stocks to collapse                                                                                                                                                                                                                                                                                                                                                                                                                                                                                                                                                                                                                                                                                                                                                                                                   \\
            \textbf{Context}: Ctenophores may be abundant during the summer months in some coastal locations, but in other places they are uncommon and difficult to find. In bays where they occur in very high numbers, predation by ctenophores may control the populations of small zooplanktonic organisms such as copepods, which might otherwise wipe out the phytoplankton (planktonic plants), which are a vital part of marine food chains. One ctenophore, Mnemiopsis, has accidentally been introduced into the Black Sea, where it is blamed for causing fish stocks to collapse by eating both fish larvae and organisms that would otherwise have fed the fish. The situation was aggravated by other factors, such as over-fishing and long-term environmental changes that promoted the growth of the Mnemiopsis population. The later accidental introduction of Beroe helped to mitigate the problem, as Beroe preys on other ctenophores. \\
        \end{tabular}
    };
    \label{fig:ex-5726431d271a42140099d7f8}
\end{figure*}

\begin{figure*}[ht]
    \center
    \tikz\node[draw=black!40!lightblue,inner sep=1pt,line width=0.3mm,rounded corners=0.1cm]{
        \begin{tabular}{p{.95\textwidth}}
            \textbf{\discability{}}: -7.43 \textbf{\diff{}}: -0.209 \textbf{Feasibility}: 0.483 \textbf{Mean Accuracy}: 0.391                                                                                                                                                                                                                                                                                                                                                                                                                                                                                                                                                                                                                                                                                                                                                                                                                                                                                                                                                                                                                                                                                                                                         \\
            \textbf{Validity}: flawed \textbf{Reason}: answer\_partially\_correct                                                                                                                                                                                                                                                                                                                                                                                                                                                                                                                                                                                                                                                                                                                                                                                                                                                                                                                                                                                                                                                                                                                                                                                     \\
            \textbf{Wikipedia Page}: \underline{Civil\_disobedience} \textbf{Question ID}: 5728e5224b864d1900165033                                                                                                                                                                                                                                                                                                                                                                                                                                                                                                                                                                                                                                                                                                                                                                                                                                                                                                                                                                                                                                                                                                                                                   \\
            \textbf{Question}: What is an example of illegal disobedience?                                                                                                                                                                                                                                                                                                                                                                                                                                                                                                                                                                                                                                                                                                                                                                                                                                                                                                                                                                                                                                                                                                                                                                                            \\
            \textbf{Official Answer}: trespassing at a nuclear-missile installation \textbf{|} symbolic illegal protests \textbf{|} trespassing at a nuclear-missile installation \textbf{|} the proprietors of illegal medical cannabis dispensaries \textbf{|} trespassing at a nuclear-missile installation                                                                                                                                                                                                                                                                                                                                                                                                                                                                                                                                                                                                                                                                                                                                                                                                                                                                                                                                                        \\
            \textbf{Context}: Civil disobedients have chosen a variety of different illegal acts. Bedau writes, "There is a whole class of acts, undertaken in the name of civil disobedience, which, even if they were widely practiced, would in themselves constitute hardly more than a nuisance (e.g. trespassing at a nuclear-missile installation)...Such acts are often just a harassment and, at least to the bystander, somewhat inane...The remoteness of the connection between the disobedient act and the objectionable law lays such acts open to the charge of ineffectiveness and absurdity." Bedau also notes, though, that the very harmlessness of such entirely symbolic illegal protests toward public policy goals may serve a propaganda purpose. Some civil disobedients, such as the proprietors of illegal medical cannabis dispensaries and Voice in the Wilderness, which brought medicine to Iraq without the permission of the U.S. Government, directly achieve a desired social goal (such as the provision of medication to the sick) while openly breaking the law. Julia Butterfly Hill lived in Luna, a 180-foot (55 m)-tall, 600-year-old California Redwood tree for 738 days, successfully preventing it from being cut down. \\
        \end{tabular}
    };
    \label{fig:ex-5728e5224b864d1900165033}
\end{figure*}

\begin{figure*}[ht]
    \center
    \tikz\node[draw=black!40!lightblue,inner sep=1pt,line width=0.3mm,rounded corners=0.1cm]{
        \begin{tabular}{p{.95\textwidth}}
            \textbf{\discability{}}: -7.30 \textbf{\diff{}}: -0.578 \textbf{Feasibility}: 0.441 \textbf{Mean Accuracy}: 0.335                                                                                                                                                                                                                                                                                                                                                                                                                                                                                                                                                                                                                                                                                                                               \\
            \textbf{Validity}: flawed \textbf{Reason}: incomplete\_answer                                                                                                                                                                                                                                                                                                                                                                                                                                                                                                                                                                                                                                                                                                                                                                                   \\
            \textbf{Wikipedia Page}: \underline{Civil\_disobedience} \textbf{Question ID}: 5728f50baf94a219006a9e55                                                                                                                                                                                                                                                                                                                                                                                                                                                                                                                                                                                                                                                                                                                                         \\
            \textbf{Question}: What way do some people perform civil disobedience in a constructive way?                                                                                                                                                                                                                                                                                                                                                                                                                                                                                                                                                                                                                                                                                                                                                    \\
            \textbf{Official Answer}: defiant speech \textbf{|} defiant speech \textbf{|} allocution \textbf{|} defiant speech \textbf{|} defiant speech \textbf{|} defiant speech                                                                                                                                                                                                                                                                                                                                                                                                                                                                                                                                                                                                                                                                          \\
            \textbf{Context}: Some civil disobedience defendants choose to make a defiant speech, or a speech explaining their actions, in allocution. In U.S. v. Burgos-Andujar, a defendant who was involved in a movement to stop military exercises by trespassing on U.S. Navy property argued to the court in allocution that "the ones who are violating the greater law are the members of the Navy". As a result, the judge increased her sentence from 40 to 60 days. This action was upheld because, according to the U.S. Court of Appeals for the First Circuit, her statement suggested a lack of remorse, an attempt to avoid responsibility for her actions, and even a likelihood of repeating her illegal actions. Some of the other allocution speeches given by the protesters complained about mistreatment from government officials. \\
        \end{tabular}
    };
    \label{fig:ex-5728f50baf94a219006a9e55}
\end{figure*}

\clearpage

\section{Discriminability Near Zero}

\begin{figure*}[ht]
    \center
    \tikz\node[draw=black!40!lightblue,inner sep=1pt,line width=0.3mm,rounded corners=0.1cm]{
        \begin{tabular}{p{.95\textwidth}}
            \textbf{\discability{}}: -1.19e-3 \textbf{\diff{}}: 3.68 \textbf{Feasibility}: 1.00 \textbf{Mean Accuracy}: 0.646                                                                                                                                                                                                                                                                                                                                                                                                                                                                                                                                                                                                                                                                                                                                                                                           \\
            \textbf{Validity}: correct \textbf{Reason}: NA                                                                                                                                                                                                                                                                                                                                                                                                                                                                                                                                                                                                                                                                                                                                                                                                                                                              \\
            \textbf{Wikipedia Page}: \underline{Oxygen} \textbf{Question ID}: 571caac55efbb31900334dc9                                                                                                                                                                                                                                                                                                                                                                                                                                                                                                                                                                                                                                                                                                                                                                                                                  \\
            \textbf{Question}: What minor amount of liquid oxygen was produced by early French experimenters?                                                                                                                                                                                                                                                                                                                                                                                                                                                                                                                                                                                                                                                                                                                                                                                                           \\
            \textbf{Official Answer}: few drops \textbf{|} a few drops \textbf{|} a few drops \textbf{|} a few drops \textbf{|} Only a few drops                                                                                                                                                                                                                                                                                                                                                                                                                                                                                                                                                                                                                                                                                                                                                                        \\
            \textbf{Context}: By the late 19th century scientists realized that air could be liquefied, and its components isolated, by compressing and cooling it. Using a cascade method, Swiss chemist and physicist Raoul Pierre Pictet evaporated liquid sulfur dioxide in order to liquefy carbon dioxide, which in turn was evaporated to cool oxygen gas enough to liquefy it. He sent a telegram on December 22, 1877 to the French Academy of Sciences in Paris announcing his discovery of liquid oxygen. Just two days later, French physicist Louis Paul Cailletet announced his own method of liquefying molecular oxygen. Only a few drops of the liquid were produced in either case so no meaningful analysis could be conducted. Oxygen was liquified in stable state for the first time on March 29, 1883 by Polish scientists from Jagiellonian University, Zygmunt Wroblewski and Karol Olszewski. \\
        \end{tabular}
    };
    \label{fig:ex-571caac55efbb31900334dc9}
\end{figure*}

\begin{figure*}[ht]
    \center
    \tikz\node[draw=black!40!lightblue,inner sep=1pt,line width=0.3mm,rounded corners=0.1cm]{
        \begin{tabular}{p{.95\textwidth}}
            \textbf{\discability{}}: -1.28e-3 \textbf{\diff{}}: 1.58 \textbf{Feasibility}: 0.0681 \textbf{Mean Accuracy}: 0.0497                                                                                                                                                                                                                                                                                                                                                                                                                                                                                                                                                                                                                                                                                                                                    \\
            \textbf{Validity}: correct \textbf{Reason}: one\_answer\_wrong                                                                                                                                                                                                                                                                                                                                                                                                                                                                                                                                                                                                                                                                                                                                                                                          \\
            \textbf{Wikipedia Page}: \underline{Geology} \textbf{Question ID}: 572663a9f1498d1400e8ddf2                                                                                                                                                                                                                                                                                                                                                                                                                                                                                                                                                                                                                                                                                                                                                             \\
            \textbf{Question}: Why is the second timeline needed?                                                                                                                                                                                                                                                                                                                                                                                                                                                                                                                                                                                                                                                                                                                                                                                                   \\
            \textbf{Official Answer}: second scale shows the most recent eon with an expanded scale \textbf{|} compresses the most recent era \textbf{|} compresses the most recent era                                                                                                                                                                                                                                                                                                                                                                                                                                                                                                                                                                                                                                                                             \\
            \textbf{Context}: The following four timelines show the geologic time scale. The first shows the entire time from the formation of the Earth to the present, but this compresses the most recent eon. Therefore, the second scale shows the most recent eon with an expanded scale. The second scale compresses the most recent era, so the most recent era is expanded in the third scale. Since the Quaternary is a very short period with short epochs, it is further expanded in the fourth scale. The second, third, and fourth timelines are therefore each subsections of their preceding timeline as indicated by asterisks. The Holocene (the latest epoch) is too small to be shown clearly on the third timeline on the right, another reason for expanding the fourth scale. The Pleistocene (P) epoch. Q stands for the Quaternary period. \\
        \end{tabular}
    };
    \label{fig:ex-572663a9f1498d1400e8ddf2}
\end{figure*}

\begin{figure*}[ht]
    \center
    \tikz\node[draw=black!40!lightblue,inner sep=1pt,line width=0.3mm,rounded corners=0.1cm]{
        \begin{tabular}{p{.95\textwidth}}
            \textbf{\discability{}}: 1.31e-3 \textbf{\diff{}}: 3.98 \textbf{Feasibility}: 1.00 \textbf{Mean Accuracy}: 0.652                                                                                                                                                                                                                                                                                                                                                                                                                                                                                                                                                                                                                                                                                              \\
            \textbf{Validity}: correct \textbf{Reason}: misleading                                                                                                                                                                                                                                                                                                                                                                                                                                                                                                                                                                                                                                                                                                                                                        \\
            \textbf{Wikipedia Page}: \underline{Warsaw} \textbf{Question ID}: 5ad4d2855b96ef001a10a1dc                                                                                                                                                                                                                                                                                                                                                                                                                                                                                                                                                                                                                                                                                                                    \\
            \textbf{Question}: What is the axis of Vistula which divides it into two parts?                                                                                                                                                                                                                                                                                                                                                                                                                                                                                                                                                                                                                                                                                                                               \\
            \textbf{Official Answer}: Not Answerable                                                                                                                                                                                                                                                                                                                                                                                                                                                                                                                                                                                                                                                                                                                                                                      \\
            \textbf{Context}: Warsaw is located on two main geomorphologic formations: the plain moraine plateau and the Vistula Valley with its asymmetrical pattern of different terraces. The Vistula River is the specific axis of Warsaw, which divides the city into two parts, left and right. The left one is situated both on the moraine plateau (10 to 25 m (32.8 to 82.0 ft) above Vistula level) and on the Vistula terraces (max. 6.5 m (21.3 ft) above Vistula level). The significant element of the relief, in this part of Warsaw, is the edge of moraine plateau called Warsaw Escarpment. It is 20 to 25 m (65.6 to 82.0 ft) high in the Old Town and Central district and about 10 m (32.8 ft) in the north and south of Warsaw. It goes through the city and plays an important role as a landmark. \\
        \end{tabular}
    };
    \label{fig:ex-5ad4d2855b96ef001a10a1dc}
\end{figure*}

\begin{figure*}[ht]
    \center
    \tikz\node[draw=black!40!lightblue,inner sep=1pt,line width=0.3mm,rounded corners=0.1cm]{
        \begin{tabular}{p{.95\textwidth}}
            \textbf{\discability{}}: 1.34e-3 \textbf{\diff{}}: -0.793 \textbf{Feasibility}: 0.793 \textbf{Mean Accuracy}: 0.404                                                                                                                                                                                                                                                                                                                                                                                   \\
            \textbf{Validity}: correct \textbf{Reason}: NA                                                                                                                                                                                                                                                                                                                                                                                                                                                        \\
            \textbf{Wikipedia Page}: \underline{Computational\_complexity\_theory} \textbf{Question ID}: 56e1a0dccd28a01900c67a2f                                                                                                                                                                                                                                                                                                                                                                                 \\
            \textbf{Question}: If two integers are multiplied and output a value, what is this expression set called?                                                                                                                                                                                                                                                                                                                                                                                             \\
            \textbf{Official Answer}: set of triples \textbf{|} triple \textbf{|} the set of triples (a, b, c) such that the relation a x b = c holds                                                                                                                                                                                                                                                                                                                                                             \\
            \textbf{Context}: It is tempting to think that the notion of function problems is much richer than the notion of decision problems. However, this is not really the case, since function problems can be recast as decision problems. For example, the multiplication of two integers can be expressed as the set of triples (a, b, c) such that the relation a x b = c holds. Deciding whether a given triple is a member of this set corresponds to solving the problem of multiplying two numbers. \\
        \end{tabular}
    };
    \label{fig:ex-56e1a0dccd28a01900c67a2f}
\end{figure*}

\begin{figure*}[ht]
    \center
    \tikz\node[draw=black!40!lightblue,inner sep=1pt,line width=0.3mm,rounded corners=0.1cm]{
        \begin{tabular}{p{.95\textwidth}}
            \textbf{\discability{}}: 1.38e-3 \textbf{\diff{}}: -0.836 \textbf{Feasibility}: 0.484 \textbf{Mean Accuracy}: 0.224                                                                                                                                                                                                                                                                                                                                                                                                                                                                                                                                                                                                                                                                                                                                                                                                                                                                                                                                                                                                                                                                                                                                                                                                                                                                                                                                                                                                                                                                                                                                                                                                                                                                                                                                                                                                                                                                                     \\
            \textbf{Validity}: correct \textbf{Reason}: NA                                                                                                                                                                                                                                                                                                                                                                                                                                                                                                                                                                                                                                                                                                                                                                                                                                                                                                                                                                                                                                                                                                                                                                                                                                                                                                                                                                                                                                                                                                                                                                                                                                                                                                                                                                                                                                                                                                                                                          \\
            \textbf{Wikipedia Page}: \underline{European\_Union\_law} \textbf{Question ID}: 5725ca4389a1e219009abeb6                                                                                                                                                                                                                                                                                                                                                                                                                                                                                                                                                                                                                                                                                                                                                                                                                                                                                                                                                                                                                                                                                                                                                                                                                                                                                                                                                                                                                                                                                                                                                                                                                                                                                                                                                                                                                                                                                                \\
            \textbf{Question}: What powers does the Court of Justice of the European Union have in regards to treaties?                                                                                                                                                                                                                                                                                                                                                                                                                                                                                                                                                                                                                                                                                                                                                                                                                                                                                                                                                                                                                                                                                                                                                                                                                                                                                                                                                                                                                                                                                                                                                                                                                                                                                                                                                                                                                                                                                             \\
            \textbf{Official Answer}: can interpret the Treaties, but it cannot rule on their validity \textbf{|} The Court of Justice of the European Union can interpret the Treaties \textbf{|} The Court of Justice of the European Union can interpret the Treaties \textbf{|} The Court of Justice of the European Union can interpret the Treaties                                                                                                                                                                                                                                                                                                                                                                                                                                                                                                                                                                                                                                                                                                                                                                                                                                                                                                                                                                                                                                                                                                                                                                                                                                                                                                                                                                                                                                                                                                                                                                                                                                                           \\
            \textbf{Context}: The primary law of the EU consists mainly of the founding treaties, the "core" treaties being the Treaty on European Union (TEU) and the Treaty on the Functioning of the European Union (TFEU). The Treaties contain formal and substantive provisions, which frame policies of the European Union institutions and determine the division of competences between the European Union and its member states. The TEU establishes that European Union law applies to the metropolitan territories of the member states, as well as certain islands and overseas territories, including Madeira, the Canary Islands and the French overseas departments. European Union law also applies in territories where a member state is responsible for external relations, for example Gibraltar and the Aland islands. The TEU allows the European Council to make specific provisions for regions, as for example done for customs matters in Gibraltar and Saint-Pierre-et-Miquelon. The TEU specifically excludes certain regions, for example the Faroe Islands, from the jurisdiction of European Union law. Treaties apply as soon as they enter into force, unless stated otherwise, and are generally concluded for an unlimited period. The TEU provides that commitments entered into by the member states between themselves before the treaty was signed no longer apply.[vague] All EU member states are regarded as subject to the general obligation of the principle of cooperation, as stated in the TEU, whereby member states are obliged not to take measure which could jeopardise the attainment of the TEU objectives. The Court of Justice of the European Union can interpret the Treaties, but it cannot rule on their validity, which is subject to international law. Individuals may rely on primary law in the Court of Justice of the European Union if the Treaty provisions have a direct effect and they are sufficiently clear, precise and unconditional. \\
        \end{tabular}
    };
    \label{fig:ex-5725ca4389a1e219009abeb6}
\end{figure*}

\begin{figure*}[ht]
    \center
    \tikz\node[draw=black!40!lightblue,inner sep=1pt,line width=0.3mm,rounded corners=0.1cm]{
        \begin{tabular}{p{.95\textwidth}}
            \textbf{\discability{}}: 2.04e-3 \textbf{\diff{}}: -0.886 \textbf{Feasibility}: 0.194 \textbf{Mean Accuracy}: 0.0994                                                                                                                                                                                                                                                                                                                                                                                                                                                                                                                                                                                                                                                                                                                                                                                                                                                                                                                                                                                                                                   \\
            \textbf{Validity}: correct \textbf{Reason}: NA                                                                                                                                                                                                                                                                                                                                                                                                                                                                                                                                                                                                                                                                                                                                                                                                                                                                                                                                                                                                                                                                                                         \\
            \textbf{Wikipedia Page}: \underline{Civil\_disobedience} \textbf{Question ID}: 5728e07e3acd2414000e00eb                                                                                                                                                                                                                                                                                                                                                                                                                                                                                                                                                                                                                                                                                                                                                                                                                                                                                                                                                                                                                                                \\
            \textbf{Question}: What is the goal of individual civil disobedience?                                                                                                                                                                                                                                                                                                                                                                                                                                                                                                                                                                                                                                                                                                                                                                                                                                                                                                                                                                                                                                                                                  \\
            \textbf{Official Answer}: render certain laws ineffective \textbf{|} to render certain laws ineffective, to cause their repeal, or to exert pressure to get one's political wishes on some other issue \textbf{|} render certain laws ineffective, to cause their repeal \textbf{|} t to render certain laws ineffective, to cause their repeal, or to exert pressure to get one's political wishes on some other issue \textbf{|} to render certain laws ineffective,                                                                                                                                                                                                                                                                                                                                                                                                                                                                                                                                                                                                                                                                                 \\
            \textbf{Context}: Non-revolutionary civil disobedience is a simple disobedience of laws on the grounds that they are judged "wrong" by an individual conscience, or as part of an effort to render certain laws ineffective, to cause their repeal, or to exert pressure to get one's political wishes on some other issue. Revolutionary civil disobedience is more of an active attempt to overthrow a government (or to change cultural traditions, social customs, religious beliefs, etc...revolution doesn't have to be political, i.e. "cultural revolution", it simply implies sweeping and widespread change to a section of the social fabric). Gandhi's acts have been described as revolutionary civil disobedience. It has been claimed that the Hungarians under Ferenc Deak directed revolutionary civil disobedience against the Austrian government. Thoreau also wrote of civil disobedience accomplishing "peaceable revolution." Howard Zinn, Harvey Wheeler, and others have identified the right espoused in The Declaration of Independence to "alter or abolish" an unjust government to be a principle of civil disobedience. \\
        \end{tabular}
    };
    \label{fig:ex-5728e07e3acd2414000e00eb}
\end{figure*}

\begin{figure*}[ht]
    \center
    \tikz\node[draw=black!40!lightblue,inner sep=1pt,line width=0.3mm,rounded corners=0.1cm]{
        \begin{tabular}{p{.95\textwidth}}
            \textbf{\discability{}}: 3.95e-3 \textbf{\diff{}}: -0.717 \textbf{Feasibility}: 0.403 \textbf{Mean Accuracy}: 0.205                                                                                                                                                                                                                                                                                                                                                                                                                                             \\
            \textbf{Validity}: correct \textbf{Reason}: ambiguous                                                                                                                                                                                                                                                                                                                                                                                                                                                                                                           \\
            \textbf{Wikipedia Page}: \underline{Amazon\_rainforest} \textbf{Question ID}: 5725c071271a42140099d12b                                                                                                                                                                                                                                                                                                                                                                                                                                                          \\
            \textbf{Question}: Where did it join in the direction of its flow?                                                                                                                                                                                                                                                                                                                                                                                                                                                                                              \\
            \textbf{Official Answer}: joining the easterly flow toward the Atlantic. \textbf{|} the easterly flow \textbf{|} easterly                                                                                                                                                                                                                                                                                                                                                                                                                                       \\
            \textbf{Context}: During the mid-Eocene, it is believed that the drainage basin of the Amazon was split along the middle of the continent by the Purus Arch. Water on the eastern side flowed toward the Atlantic, while to the west water flowed toward the Pacific across the Amazonas Basin. As the Andes Mountains rose, however, a large basin was created that enclosed a lake; now known as the Solimoes Basin. Within the last 5-10 million years, this accumulating water broke through the Purus Arch, joining the easterly flow toward the Atlantic. \\
        \end{tabular}
    };
    \label{fig:ex-5725c071271a42140099d12b}
\end{figure*}

\begin{figure*}[ht]
    \center
    \tikz\node[draw=black!40!lightblue,inner sep=1pt,line width=0.3mm,rounded corners=0.1cm]{
        \begin{tabular}{p{.95\textwidth}}
            \textbf{\discability{}}: 4.01e-3 \textbf{\diff{}}: -1.48 \textbf{Feasibility}: 1.00 \textbf{Mean Accuracy}: 0.553                                                                                                                                                                                                                                                                                                                                                                                                                                                                                                                                                                                                                                                                                                                                                                                                                                                                                                                             \\
            \textbf{Validity}: correct \textbf{Reason}: NA                                                                                                                                                                                                                                                                                                                                                                                                                                                                                                                                                                                                                                                                                                                                                                                                                                                                                                                                                                                                \\
            \textbf{Wikipedia Page}: \underline{Computational\_complexity\_theory} \textbf{Question ID}: 56e1ec83cd28a01900c67c0c                                                                                                                                                                                                                                                                                                                                                                                                                                                                                                                                                                                                                                                                                                                                                                                                                                                                                                                         \\
            \textbf{Question}: That there currently exists no known integer factorization problem underpins what commonly used system?                                                                                                                                                                                                                                                                                                                                                                                                                                                                                                                                                                                                                                                                                                                                                                                                                                                                                                                    \\
            \textbf{Official Answer}: modern cryptographic systems \textbf{|} modern cryptographic systems \textbf{|} RSA algorithm                                                                                                                                                                                                                                                                                                                                                                                                                                                                                                                                                                                                                                                                                                                                                                                                                                                                                                                       \\
            \textbf{Context}: The integer factorization problem is the computational problem of determining the prime factorization of a given integer. Phrased as a decision problem, it is the problem of deciding whether the input has a factor less than k. No efficient integer factorization algorithm is known, and this fact forms the basis of several modern cryptographic systems, such as the RSA algorithm. The integer factorization problem is in NP and in co-NP (and even in UP and co-UP). If the problem is NP-complete, the polynomial time hierarchy will collapse to its first level (i.e., NP will equal co-NP). The best known algorithm for integer factorization is the general number field sieve, which takes time O(e(64/9)1/3(n.log 2)1/3(log (n.log 2))2/3) to factor an n-bit integer. However, the best known quantum algorithm for this problem, Shor's algorithm, does run in polynomial time. Unfortunately, this fact doesn't say much about where the problem lies with respect to non-quantum complexity classes. \\
        \end{tabular}
    };
    \label{fig:ex-56e1ec83cd28a01900c67c0c}
\end{figure*}

\begin{figure*}[ht]
    \center
    \tikz\node[draw=black!40!lightblue,inner sep=1pt,line width=0.3mm,rounded corners=0.1cm]{
        \begin{tabular}{p{.95\textwidth}}
            \textbf{\discability{}}: -4.19e-3 \textbf{\diff{}}: -1.13 \textbf{Feasibility}: 0.990 \textbf{Mean Accuracy}: 0.491                                                                                                                                                                                                                                                                                                                                                                                                                                                                                                         \\
            \textbf{Validity}: correct \textbf{Reason}: incomplete                                                                                                                                                                                                                                                                                                                                                                                                                                                                                                                                                                      \\
            \textbf{Wikipedia Page}: \underline{Rhine} \textbf{Question ID}: 572ffb02b2c2fd14005686b8                                                                                                                                                                                                                                                                                                                                                                                                                                                                                                                                   \\
            \textbf{Question}: What elements from the rift system in the Alpine orogeny in Southwest Germany?                                                                                                                                                                                                                                                                                                                                                                                                                                                                                                                           \\
            \textbf{Official Answer}: Upper Rhine Graben \textbf{|} Upper Rhine Graben \textbf{|} Upper Rhine Graben                                                                                                                                                                                                                                                                                                                                                                                                                                                                                                                    \\
            \textbf{Context}: From the Eocene onwards, the ongoing Alpine orogeny caused a N-S rift system to develop in this zone. The main elements of this rift are the Upper Rhine Graben, in southwest Germany and eastern France and the Lower Rhine Embayment, in northwest Germany and the southeastern Netherlands. By the time of the Miocene, a river system had developed in the Upper Rhine Graben, that continued northward and is considered the first Rhine river. At that time, it did not yet carry discharge from the Alps; instead, the watersheds of the Rhone and Danube drained the northern flanks of the Alps. \\
        \end{tabular}
    };
    \label{fig:ex-572ffb02b2c2fd14005686b8}
\end{figure*}

\begin{figure*}[ht]
    \center
    \tikz\node[draw=black!40!lightblue,inner sep=1pt,line width=0.3mm,rounded corners=0.1cm]{
        \begin{tabular}{p{.95\textwidth}}
            \textbf{\discability{}}: 5.10e-3 \textbf{\diff{}}: 4.55 \textbf{Feasibility}: 1.00 \textbf{Mean Accuracy}: 0.658                                                                                                                                                                                                                                                                                                                      \\
            \textbf{Validity}: correct \textbf{Reason}: NA                                                                                                                                                                                                                                                                                                                                                                                        \\
            \textbf{Wikipedia Page}: \underline{Computational\_complexity\_theory} \textbf{Question ID}: 56e1ded7cd28a01900c67bd5                                                                                                                                                                                                                                                                                                                 \\
            \textbf{Question}: What is the name for a problem that meets Ladner's assertion?                                                                                                                                                                                                                                                                                                                                                      \\
            \textbf{Official Answer}: NP-intermediate problems \textbf{|} NP-intermediate problems \textbf{|} NP-intermediate                                                                                                                                                                                                                                                                                                                     \\
            \textbf{Context}: It was shown by Ladner that if P [?] NP then there exist problems in NP that are neither in P nor NP-complete. Such problems are called NP-intermediate problems. The graph isomorphism problem, the discrete logarithm problem and the integer factorization problem are examples of problems believed to be NP-intermediate. They are some of the very few NP problems not known to be in P or to be NP-complete. \\
        \end{tabular}
    };
    \label{fig:ex-56e1ded7cd28a01900c67bd5}
\end{figure*}

\clearpage

\section{High Discriminability}

\begin{figure*}[ht]
    \center
    \tikz\node[draw=black!40!lightblue,inner sep=1pt,line width=0.3mm,rounded corners=0.1cm]{
        \begin{tabular}{p{.95\textwidth}}
            \textbf{\discability{}}: 12.6 \textbf{\diff{}}: -1.54 \textbf{Feasibility}: 1.00 \textbf{Mean Accuracy}: 0.832                                                                                                                                                                                                                                                                                                                                                                                                                                                                                                           \\
            \textbf{Validity}: flawed \textbf{Reason}: answer\_partially\_correct                                                                                                                                                                                                                                                                                                                                                                                                                                                                                                                                                    \\
            \textbf{Wikipedia Page}: \underline{Computational\_complexity\_theory} \textbf{Question ID}: 56e1febfe3433e1400423239                                                                                                                                                                                                                                                                                                                                                                                                                                                                                                    \\
            \textbf{Question}: How quickly can an algorithm solve an NP-complete knapsack problem?                                                                                                                                                                                                                                                                                                                                                                                                                                                                                                                                   \\
            \textbf{Official Answer}: in less than quadratic time \textbf{|} less than quadratic time \textbf{|} less than quadratic time                                                                                                                                                                                                                                                                                                                                                                                                                                                                                            \\
            \textbf{Context}: What intractability means in practice is open to debate. Saying that a problem is not in P does not imply that all large cases of the problem are hard or even that most of them are. For example, the decision problem in Presburger arithmetic has been shown not to be in P, yet algorithms have been written that solve the problem in reasonable times in most cases. Similarly, algorithms can solve the NP-complete knapsack problem over a wide range of sizes in less than quadratic time and SAT solvers routinely handle large instances of the NP-complete Boolean satisfiability problem. \\
        \end{tabular}
    };
    \label{fig:ex-56e1febfe3433e1400423239}
\end{figure*}

\begin{figure*}[ht]
    \center
    \tikz\node[draw=black!40!lightblue,inner sep=1pt,line width=0.3mm,rounded corners=0.1cm]{
        \begin{tabular}{p{.95\textwidth}}
            \textbf{\discability{}}: 12.6 \textbf{\diff{}}: -1.57 \textbf{Feasibility}: 1.00 \textbf{Mean Accuracy}: 0.845                                                                                                                                                                                                                                                                                                                                                                                                                                                                                                                                                                                                                                                                                                                                                                                                                                              \\
            \textbf{Validity}: correct \textbf{Reason}: NA                                                                                                                                                                                                                                                                                                                                                                                                                                                                                                                                                                                                                                                                                                                                                                                                                                                                                                              \\
            \textbf{Wikipedia Page}: \underline{Force} \textbf{Question ID}: 573784fa1c45671900574483                                                                                                                                                                                                                                                                                                                                                                                                                                                                                                                                                                                                                                                                                                                                                                                                                                                                   \\
            \textbf{Question}: Who identified gravity as a force?                                                                                                                                                                                                                                                                                                                                                                                                                                                                                                                                                                                                                                                                                                                                                                                                                                                                                                       \\
            \textbf{Official Answer}: Isaac Newton \textbf{|} Isaac Newton \textbf{|} Isaac Newton \textbf{|} Isaac Newton                                                                                                                                                                                                                                                                                                                                                                                                                                                                                                                                                                                                                                                                                                                                                                                                                                              \\
            \textbf{Context}: What we now call gravity was not identified as a universal force until the work of Isaac Newton. Before Newton, the tendency for objects to fall towards the Earth was not understood to be related to the motions of celestial objects. Galileo was instrumental in describing the characteristics of falling objects by determining that the acceleration of every object in free-fall was constant and independent of the mass of the object. Today, this acceleration due to gravity towards the surface of the Earth is usually designated as  and has a magnitude of about 9.81 meters per second squared (this measurement is taken from sea level and may vary depending on location), and points toward the center of the Earth. This observation means that the force of gravity on an object at the Earth's surface is directly proportional to the object's mass. Thus an object that has a mass of  will experience a force: \\
        \end{tabular}
    };
    \label{fig:ex-573784fa1c45671900574483}
\end{figure*}

\begin{figure*}[ht]
    \center
    \tikz\node[draw=black!40!lightblue,inner sep=1pt,line width=0.3mm,rounded corners=0.1cm]{
        \begin{tabular}{p{.95\textwidth}}
            \textbf{\discability{}}: 12.5 \textbf{\diff{}}: -0.948 \textbf{Feasibility}: 1.00 \textbf{Mean Accuracy}: 0.472                                                                                                                                                                                                                                                                                                                                                                                                                                                                                                \\
            \textbf{Validity}: correct \textbf{Reason}: NA                                                                                                                                                                                                                                                                                                                                                                                                                                                                                                                                                                 \\
            \textbf{Wikipedia Page}: \underline{Economic\_inequality} \textbf{Question ID}: 5727f16c3acd2414000df05b                                                                                                                                                                                                                                                                                                                                                                                                                                                                                                       \\
            \textbf{Question}: What did Standard \& Poor recommend to speed economy recovery?                                                                                                                                                                                                                                                                                                                                                                                                                                                                                                                              \\
            \textbf{Official Answer}: increasing access to education \textbf{|} increasing access to education \textbf{|} increasing access to education                                                                                                                                                                                                                                                                                                                                                                                                                                                                   \\
            \textbf{Context}: In 2014, economists with the Standard \& Poor's rating agency concluded that the widening disparity between the U.S.'s wealthiest citizens and the rest of the nation had slowed its recovery from the 2008-2009 recession and made it more prone to boom-and-bust cycles. To partially remedy the wealth gap and the resulting slow growth, S\&P recommended increasing access to education. It estimated that if the average United States worker had completed just one more year of school, it would add an additional \$105 billion in growth to the country's economy over five years. \\
        \end{tabular}
    };
    \label{fig:ex-5727f16c3acd2414000df05b}
\end{figure*}

\begin{figure*}[ht]
    \center
    \tikz\node[draw=black!40!lightblue,inner sep=1pt,line width=0.3mm,rounded corners=0.1cm]{
        \begin{tabular}{p{.95\textwidth}}
            \textbf{\discability{}}: 12.3 \textbf{\diff{}}: -1.56 \textbf{Feasibility}: 1.00 \textbf{Mean Accuracy}: 0.851                                                                                                                                                                                                                                                                                                                                                                                                                                                                                                                                                                                                                                                                                                                                                                                                                                                         \\
            \textbf{Validity}: correct \textbf{Reason}: NA                                                                                                                                                                                                                                                                                                                                                                                                                                                                                                                                                                                                                                                                                                                                                                                                                                                                                                                         \\
            \textbf{Wikipedia Page}: \underline{Imperialism} \textbf{Question ID}: 573098f38ab72b1400f9c5d3                                                                                                                                                                                                                                                                                                                                                                                                                                                                                                                                                                                                                                                                                                                                                                                                                                                                        \\
            \textbf{Question}: What did European empires rely on to supply them with resources?                                                                                                                                                                                                                                                                                                                                                                                                                                                                                                                                                                                                                                                                                                                                                                                                                                                                                    \\
            \textbf{Official Answer}: colonies \textbf{|} collecting resources from colonies \textbf{|} colonies \textbf{|} colonies \textbf{|} colonies                                                                                                                                                                                                                                                                                                                                                                                                                                                                                                                                                                                                                                                                                                                                                                                                                           \\
            \textbf{Context}: Europe's expansion into territorial imperialism was largely focused on economic growth by collecting resources from colonies, in combination with assuming political control by military and political means. The colonization of India in the mid-18th century offers an example of this focus: there, the "British exploited the political weakness of the Mughal state, and, while military activity was important at various times, the economic and administrative incorporation of local elites was also of crucial significance" for the establishment of control over the subcontinent's resources, markets, and manpower. Although a substantial number of colonies had been designed to provide economic profit and to ship resources to home ports in the seventeenth and eighteenth centuries, Fieldhouse suggests that in the nineteenth and twentieth centuries in places such as Africa and Asia, this idea is not necessarily valid: \\
        \end{tabular}
    };
    \label{fig:ex-573098f38ab72b1400f9c5d3}
\end{figure*}

\begin{figure*}[ht]
    \center
    \tikz\node[draw=black!40!lightblue,inner sep=1pt,line width=0.3mm,rounded corners=0.1cm]{
        \begin{tabular}{p{.95\textwidth}}
            \textbf{\discability{}}: 12.1 \textbf{\diff{}}: -1.61 \textbf{Feasibility}: 1.00 \textbf{Mean Accuracy}: 0.857                                                                              \\
            \textbf{Validity}: correct \textbf{Reason}: NA                                                                                                                                              \\
            \textbf{Wikipedia Page}: \underline{Southern\_California} \textbf{Question ID}: 5706155352bb891400689895                                                                                    \\
            \textbf{Question}: At which level of education is this sport becoming more popular?                                                                                                         \\
            \textbf{Official Answer}: high school \textbf{|} high school \textbf{|} high school                                                                                                         \\
            \textbf{Context}: Rugby is also a growing sport in southern California, particularly at the high school level, with increasing numbers of schools adding rugby as an official school sport. \\
        \end{tabular}
    };
    \label{fig:ex-5706155352bb891400689895}
\end{figure*}

\begin{figure*}[ht]
    \center
    \tikz\node[draw=black!40!lightblue,inner sep=1pt,line width=0.3mm,rounded corners=0.1cm]{
        \begin{tabular}{p{.95\textwidth}}
            \textbf{\discability{}}: 11.6 \textbf{\diff{}}: -1.42 \textbf{Feasibility}: 1.00 \textbf{Mean Accuracy}: 0.776                                                                                                                                                                                                                                                                                                                                                                                                                                                                                                                                                                                                                                                                                                                                                                                                                                                                                                                                                                                                                                                                                                                  \\
            \textbf{Validity}: correct \textbf{Reason}: NA                                                                                                                                                                                                                                                                                                                                                                                                                                                                                                                                                                                                                                                                                                                                                                                                                                                                                                                                                                                                                                                                                                                                                                                  \\
            \textbf{Wikipedia Page}: \underline{Scottish\_Parliament} \textbf{Question ID}: 572fd47fa23a5019007fca56                                                                                                                                                                                                                                                                                                                                                                                                                                                                                                                                                                                                                                                                                                                                                                                                                                                                                                                                                                                                                                                                                                                        \\
            \textbf{Question}: Who is elected at the beginning of each term?                                                                                                                                                                                                                                                                                                                                                                                                                                                                                                                                                                                                                                                                                                                                                                                                                                                                                                                                                                                                                                                                                                                                                                \\
            \textbf{Official Answer}: First Minister \textbf{|} First Minister \textbf{|} a First Minister                                                                                                                                                                                                                                                                                                                                                                                                                                                                                                                                                                                                                                                                                                                                                                                                                                                                                                                                                                                                                                                                                                                                  \\
            \textbf{Context}: The party, or parties, that hold the majority of seats in the Parliament forms the Scottish Government. In contrast to many other parliamentary systems, Parliament elects a First Minister from a number of candidates at the beginning of each parliamentary term (after a general election). Any member can put their name forward to be First Minister, and a vote is taken by all members of Parliament. Normally, the leader of the largest party is returned as First Minister, and head of the Scottish Government. Theoretically, Parliament also elects the Scottish Ministers who form the government of Scotland and sit in the Scottish cabinet, but such ministers are, in practice, appointed to their roles by the First Minister. Junior ministers, who do not attend cabinet, are also appointed to assist Scottish ministers in their departments. Most ministers and their juniors are drawn from amongst the elected MSPs, with the exception of Scotland's Chief Law Officers: the Lord Advocate and the Solicitor General. Whilst the First Minister chooses the ministers - and may decide to remove them at any time - the formal appointment or dismissal is made by the Sovereign. \\
        \end{tabular}
    };
    \label{fig:ex-572fd47fa23a5019007fca56}
\end{figure*}

\begin{figure*}[ht]
    \center
    \tikz\node[draw=black!40!lightblue,inner sep=1pt,line width=0.3mm,rounded corners=0.1cm]{
        \begin{tabular}{p{.95\textwidth}}
            \textbf{\discability{}}: 11.5 \textbf{\diff{}}: -1.53 \textbf{Feasibility}: 1.00 \textbf{Mean Accuracy}: 0.807                                                                                                                                                                                                                                                                                                                                                                                                                                               \\
            \textbf{Validity}: correct \textbf{Reason}: one\_answer\_wrong                                                                                                                                                                                                                                                                                                                                                                                                                                                                                               \\
            \textbf{Wikipedia Page}: \underline{Force} \textbf{Question ID}: 57376df3c3c5551400e51ed7                                                                                                                                                                                                                                                                                                                                                                                                                                                                    \\
            \textbf{Question}: What can keep an object from moving when it is being pushed on a surface?                                                                                                                                                                                                                                                                                                                                                                                                                                                                 \\
            \textbf{Official Answer}: static friction \textbf{|} static friction \textbf{|} friction \textbf{|} static friction \textbf{|} applied force                                                                                                                                                                                                                                                                                                                                                                                                                 \\
            \textbf{Context}: Pushing against an object on a frictional surface can result in a situation where the object does not move because the applied force is opposed by static friction, generated between the object and the table surface. For a situation with no movement, the static friction force exactly balances the applied force resulting in no acceleration. The static friction increases or decreases in response to the applied force up to an upper limit determined by the characteristics of the contact between the surface and the object. \\
        \end{tabular}
    };
    \label{fig:ex-57376df3c3c5551400e51ed7}
\end{figure*}

\begin{figure*}[ht]
    \center
    \tikz\node[draw=black!40!lightblue,inner sep=1pt,line width=0.3mm,rounded corners=0.1cm]{
        \begin{tabular}{p{.95\textwidth}}
            \textbf{\discability{}}: 10.9 \textbf{\diff{}}: -1.42 \textbf{Feasibility}: 0.694 \textbf{Mean Accuracy}: 0.497                                                                                                                                                                                                                                                                                                                                                                                                                                                                                                                                                                            \\
            \textbf{Validity}: correct \textbf{Reason}: NA                                                                                                                                                                                                                                                                                                                                                                                                                                                                                                                                                                                                                                             \\
            \textbf{Wikipedia Page}: \underline{Black\_Death} \textbf{Question ID}: 5726509bdd62a815002e815d                                                                                                                                                                                                                                                                                                                                                                                                                                                                                                                                                                                           \\
            \textbf{Question}: What did Graham Twigg publish in 1984?                                                                                                                                                                                                                                                                                                                                                                                                                                                                                                                                                                                                                                  \\
            \textbf{Official Answer}: the first major work to challenge the bubonic plague theory directly \textbf{|} the first major work to challenge the bubonic plague theory directly \textbf{|} first major work to challenge the bubonic plague theory directly,                                                                                                                                                                                                                                                                                                                                                                                                                                \\
            \textbf{Context}: The plague theory was first significantly challenged by the work of British bacteriologist J. F. D. Shrewsbury in 1970, who noted that the reported rates of mortality in rural areas during the 14th-century pandemic were inconsistent with the modern bubonic plague, leading him to conclude that contemporary accounts were exaggerations. In 1984 zoologist Graham Twigg produced the first major work to challenge the bubonic plague theory directly, and his doubts about the identity of the Black Death have been taken up by a number of authors, including Samuel K. Cohn, Jr. (2002), David Herlihy (1997), and Susan Scott and Christopher Duncan (2001). \\
        \end{tabular}
    };
    \label{fig:ex-5726509bdd62a815002e815d}
\end{figure*}

\begin{figure*}[ht]
    \center
    \tikz\node[draw=black!40!lightblue,inner sep=1pt,line width=0.3mm,rounded corners=0.1cm]{
        \begin{tabular}{p{.95\textwidth}}
            \textbf{\discability{}}: 10.7 \textbf{\diff{}}: -1.55 \textbf{Feasibility}: 1.00 \textbf{Mean Accuracy}: 0.820                                                                                                                                                                                                                                                                                                                                                                                                                                                                                                                                                                                                                                                                                       \\
            \textbf{Validity}: correct \textbf{Reason}: NA                                                                                                                                                                                                                                                                                                                                                                                                                                                                                                                                                                                                                                                                                                                                                       \\
            \textbf{Wikipedia Page}: \underline{Harvard\_University} \textbf{Question ID}: 5727b8df3acd2414000dea9a                                                                                                                                                                                                                                                                                                                                                                                                                                                                                                                                                                                                                                                                                              \\
            \textbf{Question}: What liberal succeeded Joseph Willard as president?                                                                                                                                                                                                                                                                                                                                                                                                                                                                                                                                                                                                                                                                                                                               \\
            \textbf{Official Answer}: Samuel Webber \textbf{|} Samuel Webber \textbf{|} Samuel Webber                                                                                                                                                                                                                                                                                                                                                                                                                                                                                                                                                                                                                                                                                                            \\
            \textbf{Context}: Throughout the 18th century, Enlightenment ideas of the power of reason and free will became widespread among Congregationalist ministers, putting those ministers and their congregations in tension with more traditionalist, Calvinist parties.:1-4 When the Hollis Professor of Divinity David Tappan died in 1803 and the president of Harvard Joseph Willard died a year later, in 1804, a struggle broke out over their replacements. Henry Ware was elected to the chair in 1805, and the liberal Samuel Webber was appointed to the presidency of Harvard two years later, which signaled the changing of the tide from the dominance of traditional ideas at Harvard to the dominance of liberal, Arminian ideas (defined by traditionalists as Unitarian ideas).:4-5:24 \\
        \end{tabular}
    };
    \label{fig:ex-5727b8df3acd2414000dea9a}
\end{figure*}

\begin{figure*}[ht]
    \center
    \tikz\node[draw=black!40!lightblue,inner sep=1pt,line width=0.3mm,rounded corners=0.1cm]{
        \begin{tabular}{p{.95\textwidth}}
            \textbf{\discability{}}: 10.4 \textbf{\diff{}}: -1.67 \textbf{Feasibility}: 1.00 \textbf{Mean Accuracy}: 0.876                                                                                                                                                                                                                                                                                                                                                                                                                                                                                                 \\
            \textbf{Validity}: correct \textbf{Reason}: NA                                                                                                                                                                                                                                                                                                                                                                                                                                                                                                                                                                 \\
            \textbf{Wikipedia Page}: \underline{Economic\_inequality} \textbf{Question ID}: 5729e1e36aef0514001550be                                                                                                                                                                                                                                                                                                                                                                                                                                                                                                       \\
            \textbf{Question}: What does wealth disparity make the economy more prone to?                                                                                                                                                                                                                                                                                                                                                                                                                                                                                                                                  \\
            \textbf{Official Answer}: boom-and-bust cycles \textbf{|} boom-and-bust cycles \textbf{|} boom-and-bust cycles                                                                                                                                                                                                                                                                                                                                                                                                                                                                                                 \\
            \textbf{Context}: In 2014, economists with the Standard \& Poor's rating agency concluded that the widening disparity between the U.S.'s wealthiest citizens and the rest of the nation had slowed its recovery from the 2008-2009 recession and made it more prone to boom-and-bust cycles. To partially remedy the wealth gap and the resulting slow growth, S\&P recommended increasing access to education. It estimated that if the average United States worker had completed just one more year of school, it would add an additional \$105 billion in growth to the country's economy over five years. \\
        \end{tabular}
    };
    \label{fig:ex-5729e1e36aef0514001550be}
\end{figure*}

\clearpage

\section{Low Difficulty}

\begin{figure*}[ht]
    \center
    \tikz\node[draw=black!40!lightblue,inner sep=1pt,line width=0.3mm,rounded corners=0.1cm]{
        \begin{tabular}{p{.95\textwidth}}
            \textbf{\discability{}}: 0.188 \textbf{\diff{}}: -7.33 \textbf{Feasibility}: 1.00 \textbf{Mean Accuracy}: 0.832                                                                                                                                                                                                                                                                                                                                                                                                                                                                                                                                                                            \\
            \textbf{Validity}: wrong \textbf{Reason}: is\_answerable                                                                                                                                                                                                                                                                                                                                                                                                                                                                                                                                                                                                                                   \\
            \textbf{Wikipedia Page}: \underline{Packet\_switching} \textbf{Question ID}: 5a551230134fea001a0e18d0                                                                                                                                                                                                                                                                                                                                                                                                                                                                                                                                                                                      \\
            \textbf{Question}: Late published versions were utilized by who?                                                                                                                                                                                                                                                                                                                                                                                                                                                                                                                                                                                                                           \\
            \textbf{Official Answer}: Not Answerable                                                                                                                                                                                                                                                                                                                                                                                                                                                                                                                                                                                                                                                   \\
            \textbf{Context}: DECnet is a suite of network protocols created by Digital Equipment Corporation, originally released in 1975 in order to connect two PDP-11 minicomputers. It evolved into one of the first peer-to-peer network architectures, thus transforming DEC into a networking powerhouse in the 1980s. Initially built with three layers, it later (1982) evolved into a seven-layer OSI-compliant networking protocol. The DECnet protocols were designed entirely by Digital Equipment Corporation. However, DECnet Phase II (and later) were open standards with published specifications, and several implementations were developed outside DEC, including one for Linux. \\
        \end{tabular}
    };
    \label{fig:ex-5a551230134fea001a0e18d0}
\end{figure*}

\begin{figure*}[ht]
    \center
    \tikz\node[draw=black!40!lightblue,inner sep=1pt,line width=0.3mm,rounded corners=0.1cm]{
        \begin{tabular}{p{.95\textwidth}}
            \textbf{\discability{}}: 0.233 \textbf{\diff{}}: -7.31 \textbf{Feasibility}: 1.00 \textbf{Mean Accuracy}: 0.845                                                                                                                                                                                                                                                                                                                                                                                                                                                                                                                                                                                                                                                                                                                                                                                                                                                                                                                                                                                                                                                                                                                                                                                                                                                                                                 \\
            \textbf{Validity}: correct \textbf{Reason}: NA                                                                                                                                                                                                                                                                                                                                                                                                                                                                                                                                                                                                                                                                                                                                                                                                                                                                                                                                                                                                                                                                                                                                                                                                                                                                                                                                                                  \\
            \textbf{Wikipedia Page}: \underline{Geology} \textbf{Question ID}: 5a5909b13e1742001a15cf4c                                                                                                                                                                                                                                                                                                                                                                                                                                                                                                                                                                                                                                                                                                                                                                                                                                                                                                                                                                                                                                                                                                                                                                                                                                                                                                                     \\
            \textbf{Question}: What kind of unit tends to change its volume?                                                                                                                                                                                                                                                                                                                                                                                                                                                                                                                                                                                                                                                                                                                                                                                                                                                                                                                                                                                                                                                                                                                                                                                                                                                                                                                                                \\
            \textbf{Official Answer}: Not Answerable                                                                                                                                                                                                                                                                                                                                                                                                                                                                                                                                                                                                                                                                                                                                                                                                                                                                                                                                                                                                                                                                                                                                                                                                                                                                                                                                                                        \\
            \textbf{Context}: When rock units are placed under horizontal compression, they shorten and become thicker. Because rock units, other than muds, do not significantly change in volume, this is accomplished in two primary ways: through faulting and folding. In the shallow crust, where brittle deformation can occur, thrust faults form, which cause deeper rock to move on top of shallower rock. Because deeper rock is often older, as noted by the principle of superposition, this can result in older rocks moving on top of younger ones. Movement along faults can result in folding, either because the faults are not planar or because rock layers are dragged along, forming drag folds as slip occurs along the fault. Deeper in the Earth, rocks behave plastically, and fold instead of faulting. These folds can either be those where the material in the center of the fold buckles upwards, creating "antiforms", or where it buckles downwards, creating "synforms". If the tops of the rock units within the folds remain pointing upwards, they are called anticlines and synclines, respectively. If some of the units in the fold are facing downward, the structure is called an overturned anticline or syncline, and if all of the rock units are overturned or the correct up-direction is unknown, they are simply called by the most general terms, antiforms and synforms. \\
        \end{tabular}
    };
    \label{fig:ex-5a5909b13e1742001a15cf4c}
\end{figure*}

\begin{figure*}[ht]
    \center
    \tikz\node[draw=black!40!lightblue,inner sep=1pt,line width=0.3mm,rounded corners=0.1cm]{
        \begin{tabular}{p{.95\textwidth}}
            \textbf{\discability{}}: 0.143 \textbf{\diff{}}: -7.11 \textbf{Feasibility}: 1.00 \textbf{Mean Accuracy}: 0.727                                                                                                                                                                                                                                                                                                                                                                                                                                                                                                                                       \\
            \textbf{Validity}: correct \textbf{Reason}: NA                                                                                                                                                                                                                                                                                                                                                                                                                                                                                                                                                                                                        \\
            \textbf{Wikipedia Page}: \underline{1973\_oil\_crisis} \textbf{Question ID}: 5726487b5951b619008f6ee1                                                                                                                                                                                                                                                                                                                                                                                                                                                                                                                                                 \\
            \textbf{Question}: Who wanted Israel to withdraw from its border?                                                                                                                                                                                                                                                                                                                                                                                                                                                                                                                                                                                     \\
            \textbf{Official Answer}: Ted Heath \textbf{|} Ted Heath \textbf{|} Ted Heath \textbf{|} Ted Heath                                                                                                                                                                                                                                                                                                                                                                                                                                                                                                                                                    \\
            \textbf{Context}: The embargo was not uniform across Europe. Of the nine members of the European Economic Community (EEC), the Netherlands faced a complete embargo, the UK and France received almost uninterrupted supplies (having refused to allow America to use their airfields and embargoed arms and supplies to both the Arabs and the Israelis), while the other six faced partial cutbacks. The UK had traditionally been an ally of Israel, and Harold Wilson's government supported the Israelis during the Six-Day War. His successor, Ted Heath, reversed this policy in 1970, calling for Israel to withdraw to its pre-1967 borders. \\
        \end{tabular}
    };
    \label{fig:ex-5726487b5951b619008f6ee1}
\end{figure*}

\begin{figure*}[ht]
    \center
    \tikz\node[draw=black!40!lightblue,inner sep=1pt,line width=0.3mm,rounded corners=0.1cm]{
        \begin{tabular}{p{.95\textwidth}}
            \textbf{\discability{}}: -0.207 \textbf{\diff{}}: -6.99 \textbf{Feasibility}: 1.00 \textbf{Mean Accuracy}: 0.0870                                                                                                                                                                                                                                                                                                                                                                                                                                                                                                                                                                                                                                                                                                                                                                                                                                                                                                                                                                                                                                                                                           \\
            \textbf{Validity}: correct \textbf{Reason}: NA                                                                                                                                                                                                                                                                                                                                                                                                                                                                                                                                                                                                                                                                                                                                                                                                                                                                                                                                                                                                                                                                                                                                                              \\
            \textbf{Wikipedia Page}: \underline{Southern\_California} \textbf{Question ID}: 5705e3f252bb89140068966d                                                                                                                                                                                                                                                                                                                                                                                                                                                                                                                                                                                                                                                                                                                                                                                                                                                                                                                                                                                                                                                                                                    \\
            \textbf{Question}: Which of the three heavily populated areas has the least number of inhabitants?                                                                                                                                                                                                                                                                                                                                                                                                                                                                                                                                                                                                                                                                                                                                                                                                                                                                                                                                                                                                                                                                                                          \\
            \textbf{Official Answer}: San Diego \textbf{|} the San Diego area \textbf{|} San Diego                                                                                                                                                                                                                                                                                                                                                                                                                                                                                                                                                                                                                                                                                                                                                                                                                                                                                                                                                                                                                                                                                                                      \\
            \textbf{Context}: Southern California includes the heavily built-up urban area stretching along the Pacific coast from Ventura, through the Greater Los Angeles Area and the Inland Empire, and down to Greater San Diego. Southern California's population encompasses seven metropolitan areas, or MSAs: the Los Angeles metropolitan area, consisting of Los Angeles and Orange counties; the Inland Empire, consisting of Riverside and San Bernardino counties; the San Diego metropolitan area; the Oxnard-Thousand Oaks-Ventura metropolitan area; the Santa Barbara metro area; the San Luis Obispo metropolitan area; and the El Centro area. Out of these, three are heavy populated areas: the Los Angeles area with over 12 million inhabitants, the Riverside-San Bernardino area with over four million inhabitants, and the San Diego area with over 3 million inhabitants. For CSA metropolitan purposes, the five counties of Los Angeles, Orange, Riverside, San Bernardino, and Ventura are all combined to make up the Greater Los Angeles Area with over 17.5 million people. With over 22 million people, southern California contains roughly 60 percent of California's population. \\
        \end{tabular}
    };
    \label{fig:ex-5705e3f252bb89140068966d}
\end{figure*}

\begin{figure*}[ht]
    \center
    \tikz\node[draw=black!40!lightblue,inner sep=1pt,line width=0.3mm,rounded corners=0.1cm]{
        \begin{tabular}{p{.95\textwidth}}
            \textbf{\discability{}}: 2.15 \textbf{\diff{}}: -6.86 \textbf{Feasibility}: 0.770 \textbf{Mean Accuracy}: 0.770                                                                                                                                                                                                                                                                                                                                                                                                                                              \\
            \textbf{Validity}: correct \textbf{Reason}: NA                                                                                                                                                                                                                                                                                                                                                                                                                                                                                                               \\
            \textbf{Wikipedia Page}: \underline{Packet\_switching} \textbf{Question ID}: 5726462b708984140094c118                                                                                                                                                                                                                                                                                                                                                                                                                                                        \\
            \textbf{Question}: What was the purpose of CSNET                                                                                                                                                                                                                                                                                                                                                                                                                                                                                                             \\
            \textbf{Official Answer}: to extend networking benefits, for computer science departments at academic and research institutions that could not be directly connected to ARPANET \textbf{|} to extend networking benefits \textbf{|} extend networking benefits                                                                                                                                                                                                                                                                                               \\
            \textbf{Context}: The Computer Science Network (CSNET) was a computer network funded by the U.S. National Science Foundation (NSF) that began operation in 1981. Its purpose was to extend networking benefits, for computer science departments at academic and research institutions that could not be directly connected to ARPANET, due to funding or authorization limitations. It played a significant role in spreading awareness of, and access to, national networking and was a major milestone on the path to development of the global Internet. \\
        \end{tabular}
    };
    \label{fig:ex-5726462b708984140094c118}
\end{figure*}

\begin{figure*}[ht]
    \center
    \tikz\node[draw=black!40!lightblue,inner sep=1pt,line width=0.3mm,rounded corners=0.1cm]{
        \begin{tabular}{p{.95\textwidth}}
            \textbf{\discability{}}: 0.280 \textbf{\diff{}}: -6.72 \textbf{Feasibility}: 1.00 \textbf{Mean Accuracy}: 0.801                                                                                                                                                                                                                                                                                                                                                                                                                                                                                                                                                                                                                                                                                                                                                                                                                                                                                                                                                                                                                           \\
            \textbf{Validity}: correct \textbf{Reason}: NA                                                                                                                                                                                                                                                                                                                                                                                                                                                                                                                                                                                                                                                                                                                                                                                                                                                                                                                                                                                                                                                                                            \\
            \textbf{Wikipedia Page}: \underline{Sky\_(United\_Kingdom)} \textbf{Question ID}: 570961aa200fba1400367f16                                                                                                                                                                                                                                                                                                                                                                                                                                                                                                                                                                                                                                                                                                                                                                                                                                                                                                                                                                                                                                \\
            \textbf{Question}: Who's satellites would the new free-to-air channels be broadcast from?                                                                                                                                                                                                                                                                                                                                                                                                                                                                                                                                                                                                                                                                                                                                                                                                                                                                                                                                                                                                                                                 \\
            \textbf{Official Answer}: Astra \textbf{|} Astra's \textbf{|} Astra's satellites                                                                                                                                                                                                                                                                                                                                                                                                                                                                                                                                                                                                                                                                                                                                                                                                                                                                                                                                                                                                                                                          \\
            \textbf{Context}: The service started on 1 September 1993 based on the idea from the then chief executive officer, Sam Chisholm and Rupert Murdoch, of converting the company business strategy to an entirely fee-based concept. The new package included four channels formerly available free-to-air, broadcasting on Astra's satellites, as well as introducing new channels. The service continued until the closure of BSkyB's analogue service on 27 September 2001, due to the launch and expansion of the Sky Digital platform. Some of the channels did broadcast either in the clear or soft encrypted (whereby a Videocrypt decoder was required to decode, without a subscription card) prior to their addition to the Sky Multichannels package. Within two months of the launch, BSkyB gained 400,000 new subscribers, with the majority taking at least one premium channel as well, which helped BSkyB reach 3.5 million households by mid-1994. Michael Grade criticized the operations in front of the Select Committee on National Heritage, mainly for the lack of original programming on many of the new channels. \\
        \end{tabular}
    };
    \label{fig:ex-570961aa200fba1400367f16}
\end{figure*}

\begin{figure*}[ht]
    \center
    \tikz\node[draw=black!40!lightblue,inner sep=1pt,line width=0.3mm,rounded corners=0.1cm]{
        \begin{tabular}{p{.95\textwidth}}
            \textbf{\discability{}}: 0.350 \textbf{\diff{}}: -6.68 \textbf{Feasibility}: 1.00 \textbf{Mean Accuracy}: 0.863                                                                                                                                                                                                                                                                                                                                                                                                                                                                                                                                                                                                                                                                                                       \\
            \textbf{Validity}: correct \textbf{Reason}: NA                                                                                                                                                                                                                                                                                                                                                                                                                                                                                                                                                                                                                                                                                                                                                                        \\
            \textbf{Wikipedia Page}: \underline{Immune\_system} \textbf{Question ID}: 5729ffda1d046914007796b2                                                                                                                                                                                                                                                                                                                                                                                                                                                                                                                                                                                                                                                                                                                    \\
            \textbf{Question}: Vaccination exploits what feature of the human immune system in order to be successful?                                                                                                                                                                                                                                                                                                                                                                                                                                                                                                                                                                                                                                                                                                            \\
            \textbf{Official Answer}: natural specificity of the immune system \textbf{|} natural specificity \textbf{|} the natural specificity                                                                                                                                                                                                                                                                                                                                                                                                                                                                                                                                                                                                                                                                                  \\
            \textbf{Context}: Long-term active memory is acquired following infection by activation of B and T cells. Active immunity can also be generated artificially, through vaccination. The principle behind vaccination (also called immunization) is to introduce an antigen from a pathogen in order to stimulate the immune system and develop specific immunity against that particular pathogen without causing disease associated with that organism. This deliberate induction of an immune response is successful because it exploits the natural specificity of the immune system, as well as its inducibility. With infectious disease remaining one of the leading causes of death in the human population, vaccination represents the most effective manipulation of the immune system mankind has developed. \\
        \end{tabular}
    };
    \label{fig:ex-5729ffda1d046914007796b2}
\end{figure*}

\begin{figure*}[ht]
    \center
    \tikz\node[draw=black!40!lightblue,inner sep=1pt,line width=0.3mm,rounded corners=0.1cm]{
        \begin{tabular}{p{.95\textwidth}}
            \textbf{\discability{}}: 0.195 \textbf{\diff{}}: -6.67 \textbf{Feasibility}: 1.00 \textbf{Mean Accuracy}: 0.807                                                                                                                                                                                                                                                                                                                                                                                                                                                                                                                                                                                                                                                                          \\
            \textbf{Validity}: correct \textbf{Reason}: NA                                                                                                                                                                                                                                                                                                                                                                                                                                                                                                                                                                                                                                                                                                                                           \\
            \textbf{Wikipedia Page}: \underline{Fresno,\_California} \textbf{Question ID}: 5725fe63ec44d21400f3d7de                                                                                                                                                                                                                                                                                                                                                                                                                                                                                                                                                                                                                                                                                  \\
            \textbf{Question}: In what year was the Interstate Highway System created?                                                                                                                                                                                                                                                                                                                                                                                                                                                                                                                                                                                                                                                                                                               \\
            \textbf{Official Answer}: 1950s \textbf{|} in the 1950s                                                                                                                                                                                                                                                                                                                                                                                                                                                                                                                                                                                                                                                                                                                                  \\
            \textbf{Context}: Fresno is the largest U.S. city not directly linked to an Interstate highway. When the Interstate Highway System was created in the 1950s, the decision was made to build what is now Interstate 5 on the west side of the Central Valley, and thus bypass many of the population centers in the region, instead of upgrading what is now State Route 99. Due to rapidly raising population and traffic in cities along SR 99, as well as the desirability of Federal funding, much discussion has been made to upgrade it to interstate standards and eventually incorporate it into the interstate system, most likely as Interstate 9. Major improvements to signage, lane width, median separation, vertical clearance, and other concerns are currently underway. \\
        \end{tabular}
    };
    \label{fig:ex-5725fe63ec44d21400f3d7de}
\end{figure*}

\begin{figure*}[ht]
    \center
    \tikz\node[draw=black!40!lightblue,inner sep=1pt,line width=0.3mm,rounded corners=0.1cm]{
        \begin{tabular}{p{.95\textwidth}}
            \textbf{\discability{}}: 0.253 \textbf{\diff{}}: -6.46 \textbf{Feasibility}: 1.00 \textbf{Mean Accuracy}: 0.857                                                                                                                                                                                                                                                                                                                                                                                                                                                                          \\
            \textbf{Validity}: correct \textbf{Reason}: NA                                                                                                                                                                                                                                                                                                                                                                                                                                                                                                                                           \\
            \textbf{Wikipedia Page}: \underline{Ctenophora} \textbf{Question ID}: 5725cb33271a42140099d1dd                                                                                                                                                                                                                                                                                                                                                                                                                                                                                           \\
            \textbf{Question}: Ctenophora are less complex than which other phylum?                                                                                                                                                                                                                                                                                                                                                                                                                                                                                                                  \\
            \textbf{Official Answer}: bilaterians \textbf{|} bilaterians \textbf{|} bilaterians                                                                                                                                                                                                                                                                                                                                                                                                                                                                                                      \\
            \textbf{Context}: Ctenophores form an animal phylum that is more complex than sponges, about as complex as cnidarians (jellyfish, sea anemones, etc.), and less complex than bilaterians (which include almost all other animals). Unlike sponges, both ctenophores and cnidarians have: cells bound by inter-cell connections and carpet-like basement membranes; muscles; nervous systems; and some have sensory organs. Ctenophores are distinguished from all other animals by having colloblasts, which are sticky and adhere to prey, although a few ctenophore species lack them. \\
        \end{tabular}
    };
    \label{fig:ex-5725cb33271a42140099d1dd}
\end{figure*}

\begin{figure*}[ht]
    \center
    \tikz\node[draw=black!40!lightblue,inner sep=1pt,line width=0.3mm,rounded corners=0.1cm]{
        \begin{tabular}{p{.95\textwidth}}
            \textbf{\discability{}}: -0.286 \textbf{\diff{}}: -6.42 \textbf{Feasibility}: 1.00 \textbf{Mean Accuracy}: 0.168                                                                                                                                                                                                                                                                                                                                                                                                                            \\
            \textbf{Validity}: wrong \textbf{Reason}: is\_answerable + misleading                                                                                                                                                                                                                                                                                                                                                                                                                                                                       \\
            \textbf{Wikipedia Page}: \underline{Islamism} \textbf{Question ID}: 5acfea0677cf76001a686466                                                                                                                                                                                                                                                                                                                                                                                                                                                \\
            \textbf{Question}: What Egyptian president jailed hundreds of members of the Brotherhood?                                                                                                                                                                                                                                                                                                                                                                                                                                                   \\
            \textbf{Official Answer}: Not Answerable                                                                                                                                                                                                                                                                                                                                                                                                                                                                                                    \\
            \textbf{Context}: Some elements of the Brotherhood, though perhaps against orders, did engage in violence against the government, and its founder Al-Banna was assassinated in 1949 in retaliation for the assassination of Egypt's premier Mahmud Fami Naqrashi three months earlier. The Brotherhood has suffered periodic repression in Egypt and has been banned several times, in 1948 and several years later following confrontations with Egyptian president Gamal Abdul Nasser, who jailed thousands of members for several years. \\
        \end{tabular}
    };
    \label{fig:ex-5acfea0677cf76001a686466}
\end{figure*}

\clearpage

\section{High Difficulty}

\begin{figure*}[ht]
    \center
    \tikz\node[draw=black!40!lightblue,inner sep=1pt,line width=0.3mm,rounded corners=0.1cm]{
        \begin{tabular}{p{.95\textwidth}}
            \textbf{\discability{}}: -0.162 \textbf{\diff{}}: 8.36 \textbf{Feasibility}: 1.00 \textbf{Mean Accuracy}: 0.894                                                                                                                                                                                                                                                                                                                                                                                                                                                                                                                                                                                                                                                 \\
            \textbf{Validity}: correct \textbf{Reason}: NA                                                                                                                                                                                                                                                                                                                                                                                                                                                                                                                                                                                                                                                                                                                  \\
            \textbf{Wikipedia Page}: \underline{European\_Union\_law} \textbf{Question ID}: 57268bf9dd62a815002e890a                                                                                                                                                                                                                                                                                                                                                                                                                                                                                                                                                                                                                                                        \\
            \textbf{Question}: By whom is European Law applied by?                                                                                                                                                                                                                                                                                                                                                                                                                                                                                                                                                                                                                                                                                                          \\
            \textbf{Official Answer}: the courts of member states and the Court of Justice of the European Union \textbf{|} the courts of member states and the Court of Justice of the European Union \textbf{|} the courts of member states and the Court of Justice of the European Union \textbf{|} the courts of member states and the Court of Justice of the European Union                                                                                                                                                                                                                                                                                                                                                                                          \\
            \textbf{Context}: European Union law is applied by the courts of member states and the Court of Justice of the European Union. Where the laws of member states provide for lesser rights European Union law can be enforced by the courts of member states. In case of European Union law which should have been transposed into the laws of member states, such as Directives, the European Commission can take proceedings against the member state under the Treaty on the Functioning of the European Union. The European Court of Justice is the highest court able to interpret European Union law. Supplementary sources of European Union law include case law by the Court of Justice, international law and general principles of European Union law. \\
        \end{tabular}
    };
    \label{fig:ex-57268bf9dd62a815002e890a}
\end{figure*}

\begin{figure*}[ht]
    \center
    \tikz\node[draw=black!40!lightblue,inner sep=1pt,line width=0.3mm,rounded corners=0.1cm]{
        \begin{tabular}{p{.95\textwidth}}
            \textbf{\discability{}}: -3.44 \textbf{\diff{}}: 8.23 \textbf{Feasibility}: 0.821 \textbf{Mean Accuracy}: 0.820                                                                   \\
            \textbf{Validity}: correct \textbf{Reason}: NA                                                                                                                                    \\
            \textbf{Wikipedia Page}: \underline{Oxygen} \textbf{Question ID}: 571c91c8dd7acb1400e4c10e                                                                                        \\
            \textbf{Question}: For what purpose is oxygen used by animal life?                                                                                                                \\
            \textbf{Official Answer}: cellular respiration \textbf{|} cellular respiration \textbf{|} cellular respiration \textbf{|} in cellular respiration \textbf{|} cellular respiration \\
            \textbf{Context}: The common allotrope of elemental oxygen on Earth is called dioxygen, O
            2. It is the form that is a major part of the Earth's atmosphere (see Occurrence). O2 has a bond length of 121 pm and a bond energy of 498 kJ*mol-1, which is smaller than the energy of other double bonds or pairs of single bonds in the biosphere and responsible for the exothermic reaction of O2 with any organic molecule. Due to its energy content, O2 is used by complex forms of life, such as animals, in cellular respiration (see Biological role). Other aspects of O
            2 are covered in the remainder of this article.                                                                                                                                   \\
        \end{tabular}
    };
    \label{fig:ex-571c91c8dd7acb1400e4c10e}
\end{figure*}

\begin{figure*}[ht]
    \center
    \tikz\node[draw=black!40!lightblue,inner sep=1pt,line width=0.3mm,rounded corners=0.1cm]{
        \begin{tabular}{p{.95\textwidth}}
            \textbf{\discability{}}: -0.0962 \textbf{\diff{}}: 8.18 \textbf{Feasibility}: 1.00 \textbf{Mean Accuracy}: 0.845                                                                                                                                                                                                                                                                                                                                                                                                                                                                                                                                                                                                                                                                                                                                                                                                                                                                                \\
            \textbf{Validity}: correct \textbf{Reason}: NA                                                                                                                                                                                                                                                                                                                                                                                                                                                                                                                                                                                                                                                                                                                                                                                                                                                                                                                                                  \\
            \textbf{Wikipedia Page}: \underline{Warsaw} \textbf{Question ID}: 57337f6ad058e614000b5bcc                                                                                                                                                                                                                                                                                                                                                                                                                                                                                                                                                                                                                                                                                                                                                                                                                                                                                                      \\
            \textbf{Question}: What had the number of people living in Warsaw declined to by 1945?                                                                                                                                                                                                                                                                                                                                                                                                                                                                                                                                                                                                                                                                                                                                                                                                                                                                                                          \\
            \textbf{Official Answer}: 420,000 \textbf{|} 420,000 \textbf{|} 420,000                                                                                                                                                                                                                                                                                                                                                                                                                                                                                                                                                                                                                                                                                                                                                                                                                                                                                                                         \\
            \textbf{Context}: In 1939, c. 1,300,000 people lived in Warsaw, but in 1945 - only 420,000. During the first years after the war, the population growth was c. 6\%, so shortly the city started to suffer from the lack of flats and of areas for new houses. The first remedial measure was the Warsaw area enlargement (1951) - but the city authorities were still forced to introduce residency registration limitations: only the spouses and children of the permanent residents as well as some persons of public importance (like renowned specialists) were allowed to get the registration, hence halving the population growth in the following years. It also bolstered some kind of conviction among Poles that Varsovians thought of themselves as better only because they lived in the capital. Unfortunately this belief still lives on in Poland (although not as much as it used to be) - even though since 1990 there are no limitations to residency registration anymore. \\
        \end{tabular}
    };
    \label{fig:ex-57337f6ad058e614000b5bcc}
\end{figure*}

\begin{figure*}[ht]
    \center
    \tikz\node[draw=black!40!lightblue,inner sep=1pt,line width=0.3mm,rounded corners=0.1cm]{
        \begin{tabular}{p{.95\textwidth}}
            \textbf{\discability{}}: -2.27 \textbf{\diff{}}: 8.08 \textbf{Feasibility}: 0.839 \textbf{Mean Accuracy}: 0.839                                                                                                                                                                                                                                                                                                                                                                                                                                                                                                                                                                                                                                                                                                                                                                                                                                                                                                              \\
            \textbf{Validity}: correct \textbf{Reason}: NA                                                                                                                                                                                                                                                                                                                                                                                                                                                                                                                                                                                                                                                                                                                                                                                                                                                                                                                                                                               \\
            \textbf{Wikipedia Page}: \underline{Civil\_disobedience} \textbf{Question ID}: 572822233acd2414000df557                                                                                                                                                                                                                                                                                                                                                                                                                                                                                                                                                                                                                                                                                                                                                                                                                                                                                                                      \\
            \textbf{Question}: Since Thoreau was not a well known writer what happened when he was arrested?                                                                                                                                                                                                                                                                                                                                                                                                                                                                                                                                                                                                                                                                                                                                                                                                                                                                                                                             \\
            \textbf{Official Answer}: was not covered in any newspapers \textbf{|} was not covered in any newspapers in the days, weeks and months after it happened. \textbf{|} his arrest was not covered in any newspapers \textbf{|} his arrest was not covered in any newspapers in the days, weeks and months after it happened \textbf{|} his arrest was not covered in any newspapers                                                                                                                                                                                                                                                                                                                                                                                                                                                                                                                                                                                                                                            \\
            \textbf{Context}: The earliest recorded incidents of collective civil disobedience took place during the Roman Empire[citation needed]. Unarmed Jews gathered in the streets to prevent the installation of pagan images in the Temple in Jerusalem.[citation needed][original research?] In modern times, some activists who commit civil disobedience as a group collectively refuse to sign bail until certain demands are met, such as favorable bail conditions, or the release of all the activists. This is a form of jail solidarity.[page needed] There have also been many instances of solitary civil disobedience, such as that committed by Thoreau, but these sometimes go unnoticed. Thoreau, at the time of his arrest, was not yet a well-known author, and his arrest was not covered in any newspapers in the days, weeks and months after it happened. The tax collector who arrested him rose to higher political office, and Thoreau's essay was not published until after the end of the Mexican War. \\
        \end{tabular}
    };
    \label{fig:ex-572822233acd2414000df557}
\end{figure*}

\begin{figure*}[ht]
    \center
    \tikz\node[draw=black!40!lightblue,inner sep=1pt,line width=0.3mm,rounded corners=0.1cm]{
        \begin{tabular}{p{.95\textwidth}}
            \textbf{\discability{}}: -2.74 \textbf{\diff{}}: 8.08 \textbf{Feasibility}: 0.820 \textbf{Mean Accuracy}: 0.820                                                                                                                                                                                                                                                                                                                                                                                                                                                                                                                                                                                                                                                                                                                                                                                                       \\
            \textbf{Validity}: correct \textbf{Reason}: NA                                                                                                                                                                                                                                                                                                                                                                                                                                                                                                                                                                                                                                                                                                                                                                                                                                                                        \\
            \textbf{Wikipedia Page}: \underline{Rhine} \textbf{Question ID}: 5ad2b72fd7d075001a42a023                                                                                                                                                                                                                                                                                                                                                                                                                                                                                                                                                                                                                                                                                                                                                                                                                             \\
            \textbf{Question}: When was Middle and Western Francia formed?                                                                                                                                                                                                                                                                                                                                                                                                                                                                                                                                                                                                                                                                                                                                                                                                                                                        \\
            \textbf{Official Answer}: Not Answerable                                                                                                                                                                                                                                                                                                                                                                                                                                                                                                                                                                                                                                                                                                                                                                                                                                                                              \\
            \textbf{Context}: By the 6th century, the Rhine was within the borders of Francia. In the 9th, it formed part of the border between Middle and Western Francia, but in the 10th century, it was fully within the Holy Roman Empire, flowing through Swabia, Franconia and Lower Lorraine. The mouths of the Rhine, in the county of Holland, fell to the Burgundian Netherlands in the 15th century; Holland remained contentious territory throughout the European wars of religion and the eventual collapse of the Holy Roman Empire, when the length of the Rhine fell to the First French Empire and its client states. The Alsace on the left banks of the Upper Rhine was sold to Burgundy by Archduke Sigismund of Austria in 1469 and eventually fell to France in the Thirty Years' War. The numerous historic castles in Rhineland-Palatinate attest to the importance of the river as a commercial route. \\
        \end{tabular}
    };
    \label{fig:ex-5ad2b72fd7d075001a42a023}
\end{figure*}

\begin{figure*}[ht]
    \center
    \tikz\node[draw=black!40!lightblue,inner sep=1pt,line width=0.3mm,rounded corners=0.1cm]{
        \begin{tabular}{p{.95\textwidth}}
            \textbf{\discability{}}: -0.211 \textbf{\diff{}}: 8.03 \textbf{Feasibility}: 1.00 \textbf{Mean Accuracy}: 0.839                                                                                                                                                                                                                                                                                                                                                                                                                                                                                           \\
            \textbf{Validity}: correct \textbf{Reason}: NA                                                                                                                                                                                                                                                                                                                                                                                                                                                                                                                                                            \\
            \textbf{Wikipedia Page}: \underline{Packet\_switching} \textbf{Question ID}: 5a667f50846392001a1e1c67                                                                                                                                                                                                                                                                                                                                                                                                                                                                                                     \\
            \textbf{Question}: What did Baran call his system?                                                                                                                                                                                                                                                                                                                                                                                                                                                                                                                                                        \\
            \textbf{Official Answer}: Not Answerable                                                                                                                                                                                                                                                                                                                                                                                                                                                                                                                                                                  \\
            \textbf{Context}: Starting in 1965, Donald Davies at the National Physical Laboratory, UK, independently developed the same message routing methodology as developed by Baran. He called it packet switching, a more accessible name than Baran's, and proposed to build a nationwide network in the UK. He gave a talk on the proposal in 1966, after which a person from the Ministry of Defence (MoD) told him about Baran's work. A member of Davies' team (Roger Scantlebury) met Lawrence Roberts at the 1967 ACM Symposium on Operating System Principles and suggested it for use in the ARPANET. \\
        \end{tabular}
    };
    \label{fig:ex-5a667f50846392001a1e1c67}
\end{figure*}

\begin{figure*}[ht]
    \center
    \tikz\node[draw=black!40!lightblue,inner sep=1pt,line width=0.3mm,rounded corners=0.1cm]{
        \begin{tabular}{p{.95\textwidth}}
            \textbf{\discability{}}: -0.177 \textbf{\diff{}}: 7.98 \textbf{Feasibility}: 1.00 \textbf{Mean Accuracy}: 0.894                                                                                                                                                                                                                                                                                                                                                                                                                                                                                                                                                                                                                                                                                                                                                                                                                                                                                              \\
            \textbf{Validity}: correct \textbf{Reason}: NA                                                                                                                                                                                                                                                                                                                                                                                                                                                                                                                                                                                                                                                                                                                                                                                                                                                                                                                                                               \\
            \textbf{Wikipedia Page}: \underline{Black\_Death} \textbf{Question ID}: 57264d58f1498d1400e8db7c                                                                                                                                                                                                                                                                                                                                                                                                                                                                                                                                                                                                                                                                                                                                                                                                                                                                                                             \\
            \textbf{Question}: What percent of untreated victims of the plague die within 8 days?                                                                                                                                                                                                                                                                                                                                                                                                                                                                                                                                                                                                                                                                                                                                                                                                                                                                                                                        \\
            \textbf{Official Answer}: 80 percent \textbf{|} 80 \textbf{|} 80                                                                                                                                                                                                                                                                                                                                                                                                                                                                                                                                                                                                                                                                                                                                                                                                                                                                                                                                             \\
            \textbf{Context}: Other forms of plague have been implicated by modern scientists. The modern bubonic plague has a mortality rate of 30-75\% and symptoms including fever of 38-41 degC (100-106 degF), headaches, painful aching joints, nausea and vomiting, and a general feeling of malaise. Left untreated, of those that contract the bubonic plague, 80 percent die within eight days. Pneumonic plague has a mortality rate of 90 to 95 percent. Symptoms include fever, cough, and blood-tinged sputum. As the disease progresses, sputum becomes free flowing and bright red. Septicemic plague is the least common of the three forms, with a mortality rate near 100\%. Symptoms are high fevers and purple skin patches (purpura due to disseminated intravascular coagulation). In cases of pneumonic and particularly septicemic plague, the progress of the disease is so rapid that there would often be no time for the development of the enlarged lymph nodes that were noted as buboes. \\
        \end{tabular}
    };
    \label{fig:ex-57264d58f1498d1400e8db7c}
\end{figure*}

\begin{figure*}[ht]
    \center
    \tikz\node[draw=black!40!lightblue,inner sep=1pt,line width=0.3mm,rounded corners=0.1cm]{
        \begin{tabular}{p{.95\textwidth}}
            \textbf{\discability{}}: -0.246 \textbf{\diff{}}: 7.94 \textbf{Feasibility}: 1.00 \textbf{Mean Accuracy}: 0.888                                                                                                                                                                                                                                                                                                                                                                                                                                                                                                                                                                                 \\
            \textbf{Validity}: correct \textbf{Reason}: NA                                                                                                                                                                                                                                                                                                                                                                                                                                                                                                                                                                                                                                                  \\
            \textbf{Wikipedia Page}: \underline{Sky\_(United\_Kingdom)} \textbf{Question ID}: 5a2c30dabfd06b001a5aea2c                                                                                                                                                                                                                                                                                                                                                                                                                                                                                                                                                                                      \\
            \textbf{Question}: Who reported that 17,000 customers received the service due to failed deliveries?                                                                                                                                                                                                                                                                                                                                                                                                                                                                                                                                                                                            \\
            \textbf{Official Answer}: Not Answerable                                                                                                                                                                                                                                                                                                                                                                                                                                                                                                                                                                                                                                                        \\
            \textbf{Context}: BSkyB launched its HDTV service, Sky+ HD, on 22 May 2006. Prior to its launch, BSkyB claimed that 40,000 people had registered to receive the HD service. In the week before the launch, rumours started to surface that BSkyB was having supply issues with its set top box (STB) from manufacturer Thomson. On Thursday 18 May 2006, and continuing through the weekend before launch, people were reporting that BSkyB had either cancelled or rescheduled its installation. Finally, the BBC reported that 17,000 customers had yet to receive the service due to failed deliveries. On 31 March 2012, Sky announced the total number of homes with Sky+HD was 4,222,000. \\
        \end{tabular}
    };
    \label{fig:ex-5a2c30dabfd06b001a5aea2c}
\end{figure*}

\begin{figure*}[ht]
    \center
    \tikz\node[draw=black!40!lightblue,inner sep=1pt,line width=0.3mm,rounded corners=0.1cm]{
        \begin{tabular}{p{.95\textwidth}}
            \textbf{\discability{}}: -0.224 \textbf{\diff{}}: 7.93 \textbf{Feasibility}: 1.00 \textbf{Mean Accuracy}: 0.907                                                                                                                                                                                                                                                                                                                                                                                                                                        \\
            \textbf{Validity}: correct \textbf{Reason}: NA                                                                                                                                                                                                                                                                                                                                                                                                                                                                                                         \\
            \textbf{Wikipedia Page}: \underline{Immune\_system} \textbf{Question ID}: 5ad4e61e5b96ef001a10a5b0                                                                                                                                                                                                                                                                                                                                                                                                                                                     \\
            \textbf{Question}: What is an autoimmune disease that mostly strikes men?                                                                                                                                                                                                                                                                                                                                                                                                                                                                              \\
            \textbf{Official Answer}: Not Answerable                                                                                                                                                                                                                                                                                                                                                                                                                                                                                                               \\
            \textbf{Context}: Hormones can act as immunomodulators, altering the sensitivity of the immune system. For example, female sex hormones are known immunostimulators of both adaptive and innate immune responses. Some autoimmune diseases such as lupus erythematosus strike women preferentially, and their onset often coincides with puberty. By contrast, male sex hormones such as testosterone seem to be immunosuppressive. Other hormones appear to regulate the immune system as well, most notably prolactin, growth hormone and vitamin D. \\
        \end{tabular}
    };
    \label{fig:ex-5ad4e61e5b96ef001a10a5b0}
\end{figure*}

\begin{figure*}[ht]
    \center
    \tikz\node[draw=black!40!lightblue,inner sep=1pt,line width=0.3mm,rounded corners=0.1cm]{
        \begin{tabular}{p{.95\textwidth}}
            \textbf{\discability{}}: -0.193 \textbf{\diff{}}: 7.93 \textbf{Feasibility}: 1.00 \textbf{Mean Accuracy}: 0.845                                                                                                        \\
            \textbf{Validity}: correct \textbf{Reason}: NA                                                                                                                                                                         \\
            \textbf{Wikipedia Page}: \underline{Oxygen} \textbf{Question ID}: 571cde695efbb31900334e1a                                                                                                                             \\
            \textbf{Question}: In what calcium containing body part is oxygen a part?                                                                                                                                              \\
            \textbf{Official Answer}: bones \textbf{|} bones \textbf{|} bones \textbf{|} bones \textbf{|} bones                                                                                                                    \\
            \textbf{Context}: The element is found in almost all biomolecules that are important to (or generated by) life. Only a few common complex biomolecules, such as squalene and the carotenes, contain no oxygen. Of the organic compounds with biological relevance, carbohydrates contain the largest proportion by mass of oxygen. All fats, fatty acids, amino acids, and proteins contain oxygen (due to the presence of carbonyl groups in these acids and their ester residues). Oxygen also occurs in phosphate (PO3-
            4) groups in the biologically important energy-carrying molecules ATP and ADP, in the backbone and the purines (except adenine) and pyrimidines of RNA and DNA, and in bones as calcium phosphate and hydroxylapatite. \\
        \end{tabular}
    };
    \label{fig:ex-571cde695efbb31900334e1a}
\end{figure*}

\clearpage

\section{IRT Prediction Errors}

\begin{figure*}[ht]
    \center
    \tikz\node[draw=black!40!lightblue,inner sep=1pt,line width=0.3mm,rounded corners=0.1cm]{
        \begin{tabular}{p{.95\textwidth}}
            \textbf{\discability{}}: 3.73 \textbf{\diff{}}: -1.90 \textbf{Feasibility}: 1.00 \textbf{Mean Accuracy}: 0.863                                                                                                                                                                                                                                                                                                                                                                                                                                                                                                                                                                         \\
            \textbf{Validity}: wrong \textbf{Reason}: answer\_set\_incomplete                                                                                                                                                                                                                                                                                                                                                                                                                                                                                                                                                                                                                      \\
            \textbf{Wikipedia Page}: \underline{Construction} \textbf{Question ID}: 57273f27dd62a815002e9a0c                                                                                                                                                                                                                                                                                                                                                                                                                                                                                                                                                                                       \\
            \textbf{Question}: What has a classification system for construction companies?                                                                                                                                                                                                                                                                                                                                                                                                                                                                                                                                                                                                        \\
            \textbf{Official Answer}: The Standard Industrial Classification and the newer North American Industry Classification System \textbf{|} Standard Industrial Classification \textbf{|} The Standard Industrial Classification and the newer North American Industry Classification System                                                                                                                                                                                                                                                                                                                                                                                               \\
            \textbf{Context}: The Standard Industrial Classification and the newer North American Industry Classification System have a classification system for companies that perform or otherwise engage in construction. To recognize the differences of companies in this sector, it is divided into three subsectors: building construction, heavy and civil engineering construction, and specialty trade contractors. There are also categories for construction service firms (e.g., engineering, architecture) and construction managers (firms engaged in managing construction projects without assuming direct financial responsibility for completion of the construction project). \\
        \end{tabular}
    };
    \label{fig:ex-57273f27dd62a815002e9a0c}
\end{figure*}

\begin{figure*}[ht]
    \center
    \tikz\node[draw=black!40!lightblue,inner sep=1pt,line width=0.3mm,rounded corners=0.1cm]{
        \begin{tabular}{p{.95\textwidth}}
            \textbf{\discability{}}: 2.07 \textbf{\diff{}}: -2.90 \textbf{Feasibility}: 1.00 \textbf{Mean Accuracy}: 0.944                                                                                                                                                                                                                                                                                                                                                                                                                                                                                                       \\
            \textbf{Validity}: wrong \textbf{Reason}: answer\_set\_incomplete                                                                                                                                                                                                                                                                                                                                                                                                                                                                                                                                                    \\
            \textbf{Wikipedia Page}: \underline{Pharmacy} \textbf{Question ID}: 5726e3c4dd62a815002e9407                                                                                                                                                                                                                                                                                                                                                                                                                                                                                                                         \\
            \textbf{Question}: What do clinical pharmacists often participate in?                                                                                                                                                                                                                                                                                                                                                                                                                                                                                                                                                \\
            \textbf{Official Answer}: patient care rounds drug product selection \textbf{|} interdisciplinary approach \textbf{|} patient care rounds drug product selection                                                                                                                                                                                                                                                                                                                                                                                                                                                     \\
            \textbf{Context}: Pharmacists provide direct patient care services that optimizes the use of medication and promotes health, wellness, and disease prevention. Clinical pharmacists care for patients in all health care settings, but the clinical pharmacy movement initially began inside hospitals and clinics. Clinical pharmacists often collaborate with physicians and other healthcare professionals to improve pharmaceutical care. Clinical pharmacists are now an integral part of the interdisciplinary approach to patient care. They often participate in patient care rounds drug product selection. \\
        \end{tabular}
    };
    \label{fig:ex-5726e3c4dd62a815002e9407}
\end{figure*}

\begin{figure*}[ht]
    \center
    \tikz\node[draw=black!40!lightblue,inner sep=1pt,line width=0.3mm,rounded corners=0.1cm]{
        \begin{tabular}{p{.95\textwidth}}
            \textbf{\discability{}}: 2.20 \textbf{\diff{}}: -2.36 \textbf{Feasibility}: 1.00 \textbf{Mean Accuracy}: 0.876                                                                                                                                                                                                                                                                                                                                                                                                                                                                                                                                                                                                                                                                                                                                                                                                                                                                                                                                                                                                                                                                                                                                                                                                                                                                                                                                                   \\
            \textbf{Validity}: wrong \textbf{Reason}: answer\_set\_incomplete                                                                                                                                                                                                                                                                                                                                                                                                                                                                                                                                                                                                                                                                                                                                                                                                                                                                                                                                                                                                                                                                                                                                                                                                                                                                                                                                                                                                \\
            \textbf{Wikipedia Page}: \underline{European\_Union\_law} \textbf{Question ID}: 5726a7ecf1498d1400e8e656                                                                                                                                                                                                                                                                                                                                                                                                                                                                                                                                                                                                                                                                                                                                                                                                                                                                                                                                                                                                                                                                                                                                                                                                                                                                                                                                                         \\
            \textbf{Question}: Which articles state that the member states' rights to deliver public services may not be obstructed?                                                                                                                                                                                                                                                                                                                                                                                                                                                                                                                                                                                                                                                                                                                                                                                                                                                                                                                                                                                                                                                                                                                                                                                                                                                                                                                                         \\
            \textbf{Official Answer}: Articles 106 and 107 \textbf{|} Articles 106 and 107 \textbf{|} Articles 106 and 107                                                                                                                                                                                                                                                                                                                                                                                                                                                                                                                                                                                                                                                                                                                                                                                                                                                                                                                                                                                                                                                                                                                                                                                                                                                                                                                                                   \\
            \textbf{Context}: Today, the Treaty of Lisbon prohibits anti-competitive agreements in Article 101(1), including price fixing. According to Article 101(2) any such agreements are automatically void. Article 101(3) establishes exemptions, if the collusion is for distributional or technological innovation, gives consumers a "fair share" of the benefit and does not include unreasonable restraints that risk eliminating competition anywhere (or compliant with the general principle of European Union law of proportionality). Article 102 prohibits the abuse of dominant position, such as price discrimination and exclusive dealing. Article 102 allows the European Council to regulations to govern mergers between firms (the current regulation is the Regulation 139/2004/EC). The general test is whether a concentration (i.e. merger or acquisition) with a community dimension (i.e. affects a number of EU member states) might significantly impede effective competition. Articles 106 and 107 provide that member state's right to deliver public services may not be obstructed, but that otherwise public enterprises must adhere to the same competition principles as companies. Article 107 lays down a general rule that the state may not aid or subsidise private parties in distortion of free competition and provides exemptions for charities, regional development objectives and in the event of a natural disaster. \\
        \end{tabular}
    };
    \label{fig:ex-5726a7ecf1498d1400e8e656}
\end{figure*}

\begin{figure*}[ht]
    \center
    \tikz\node[draw=black!40!lightblue,inner sep=1pt,line width=0.3mm,rounded corners=0.1cm]{
        \begin{tabular}{p{.95\textwidth}}
            \textbf{\discability{}}: 1.16 \textbf{\diff{}}: -3.92 \textbf{Feasibility}: 1.00 \textbf{Mean Accuracy}: 0.950                                                                                                                                                                                                                                                                                                                                                                                                                                                                                                                                        \\
            \textbf{Validity}: wrong \textbf{Reason}: is\_answerable                                                                                                                                                                                                                                                                                                                                                                                                                                                                                                                                                                                              \\
            \textbf{Wikipedia Page}: \underline{Computational\_complexity\_theory} \textbf{Question ID}: 5ad567055b96ef001a10adeb                                                                                                                                                                                                                                                                                                                                                                                                                                                                                                                                 \\
            \textbf{Question}: What theory is the Cobham-Edward thesis?                                                                                                                                                                                                                                                                                                                                                                                                                                                                                                                                                                                           \\
            \textbf{Official Answer}: Not Answerable                                                                                                                                                                                                                                                                                                                                                                                                                                                                                                                                                                                                              \\
            \textbf{Context}: The complexity class P is often seen as a mathematical abstraction modeling those computational tasks that admit an efficient algorithm. This hypothesis is called the Cobham-Edmonds thesis. The complexity class NP, on the other hand, contains many problems that people would like to solve efficiently, but for which no efficient algorithm is known, such as the Boolean satisfiability problem, the Hamiltonian path problem and the vertex cover problem. Since deterministic Turing machines are special non-deterministic Turing machines, it is easily observed that each problem in P is also member of the class NP. \\
        \end{tabular}
    };
    \label{fig:ex-5ad567055b96ef001a10adeb}
\end{figure*}

\begin{figure*}[ht]
    \center
    \tikz\node[draw=black!40!lightblue,inner sep=1pt,line width=0.3mm,rounded corners=0.1cm]{
        \begin{tabular}{p{.95\textwidth}}
            \textbf{\discability{}}: -1.28e-3 \textbf{\diff{}}: 1.58 \textbf{Feasibility}: 0.0681 \textbf{Mean Accuracy}: 0.0497                                                                                                                                                                                                                                                                                                                                                                                                                                                                                                                                                                                                                                                                                                                                    \\
            \textbf{Validity}: correct \textbf{Reason}: low\_probability                                                                                                                                                                                                                                                                                                                                                                                                                                                                                                                                                                                                                                                                                                                                                                                            \\
            \textbf{Wikipedia Page}: \underline{Geology} \textbf{Question ID}: 572663a9f1498d1400e8ddf2                                                                                                                                                                                                                                                                                                                                                                                                                                                                                                                                                                                                                                                                                                                                                             \\
            \textbf{Question}: Why is the second timeline needed?                                                                                                                                                                                                                                                                                                                                                                                                                                                                                                                                                                                                                                                                                                                                                                                                   \\
            \textbf{Official Answer}: second scale shows the most recent eon with an expanded scale \textbf{|} compresses the most recent era \textbf{|} compresses the most recent era                                                                                                                                                                                                                                                                                                                                                                                                                                                                                                                                                                                                                                                                             \\
            \textbf{Context}: The following four timelines show the geologic time scale. The first shows the entire time from the formation of the Earth to the present, but this compresses the most recent eon. Therefore, the second scale shows the most recent eon with an expanded scale. The second scale compresses the most recent era, so the most recent era is expanded in the third scale. Since the Quaternary is a very short period with short epochs, it is further expanded in the fourth scale. The second, third, and fourth timelines are therefore each subsections of their preceding timeline as indicated by asterisks. The Holocene (the latest epoch) is too small to be shown clearly on the third timeline on the right, another reason for expanding the fourth scale. The Pleistocene (P) epoch. Q stands for the Quaternary period. \\
        \end{tabular}
    };
    \label{fig:ex-572663a9f1498d1400e8ddf2}
\end{figure*}

\begin{figure*}[ht]
    \center
    \tikz\node[draw=black!40!lightblue,inner sep=1pt,line width=0.3mm,rounded corners=0.1cm]{
        \begin{tabular}{p{.95\textwidth}}
            \textbf{\discability{}}: -0.454 \textbf{\diff{}}: 6.59 \textbf{Feasibility}: 1.00 \textbf{Mean Accuracy}: 0.963                                                                                                                                                                                                                                                                                                                                                                                                                                                                                                                                                                                                                                                                                                                                                                                                                                                                     \\
            \textbf{Validity}: wrong \textbf{Reason}: answer\_set\_incomplete                                                                                                                                                                                                                                                                                                                                                                                                                                                                                                                                                                                                                                                                                                                                                                                                                                                                                                                   \\
            \textbf{Wikipedia Page}: \underline{Southern\_California} \textbf{Question ID}: 5705f09e75f01819005e77a4                                                                                                                                                                                                                                                                                                                                                                                                                                                                                                                                                                                                                                                                                                                                                                                                                                                                            \\
            \textbf{Question}: Other than land laws, what else were the Californios dissatisfied with?                                                                                                                                                                                                                                                                                                                                                                                                                                                                                                                                                                                                                                                                                                                                                                                                                                                                                          \\
            \textbf{Official Answer}: inequitable taxes \textbf{|} inequitable taxes \textbf{|} inequitable taxes                                                                                                                                                                                                                                                                                                                                                                                                                                                                                                                                                                                                                                                                                                                                                                                                                                                                               \\
            \textbf{Context}: Subsequently, Californios (dissatisfied with inequitable taxes and land laws) and pro-slavery southerners in the lightly populated "Cow Counties" of southern California attempted three times in the 1850s to achieve a separate statehood or territorial status separate from Northern California. The last attempt, the Pico Act of 1859, was passed by the California State Legislature and signed by the State governor John B. Weller. It was approved overwhelmingly by nearly 75\% of voters in the proposed Territory of Colorado. This territory was to include all the counties up to the then much larger Tulare County (that included what is now Kings, most of Kern, and part of Inyo counties) and San Luis Obispo County. The proposal was sent to Washington, D.C. with a strong advocate in Senator Milton Latham. However, the secession crisis following the election of Abraham Lincoln in 1860 led to the proposal never coming to a vote. \\
        \end{tabular}
    };
    \label{fig:ex-5705f09e75f01819005e77a4}
\end{figure*}

\begin{figure*}[ht]
    \center
    \tikz\node[draw=black!40!lightblue,inner sep=1pt,line width=0.3mm,rounded corners=0.1cm]{
        \begin{tabular}{p{.95\textwidth}}
            \textbf{\discability{}}: 2.28 \textbf{\diff{}}: -1.94 \textbf{Feasibility}: 1.00 \textbf{Mean Accuracy}: 0.832                                                                                                                                                                                                                                                                                                                                                                                                                                                                                                                                                             \\
            \textbf{Validity}: correct \textbf{Reason}: high\_probability + unanswerable                                                                                                                                                                                                                                                                                                                                                                                                                                                                                                                                                                                               \\
            \textbf{Wikipedia Page}: \underline{Rhine} \textbf{Question ID}: 5ad2990cd7d075001a429b61                                                                                                                                                                                                                                                                                                                                                                                                                                                                                                                                                                                  \\
            \textbf{Question}: What captured the Vosges Mountains?                                                                                                                                                                                                                                                                                                                                                                                                                                                                                                                                                                                                                     \\
            \textbf{Official Answer}: Not Answerable                                                                                                                                                                                                                                                                                                                                                                                                                                                                                                                                                                                                                                   \\
            \textbf{Context}: Through stream capture, the Rhine extended its watershed southward. By the Pliocene period, the Rhine had captured streams down to the Vosges Mountains, including the Mosel, the Main and the Neckar. The northern Alps were then drained by the Rhone. By the early Pleistocene period, the Rhine had captured most of its current Alpine watershed from the Rhone, including the Aar. Since that time, the Rhine has added the watershed above Lake Constance (Vorderrhein, Hinterrhein, Alpenrhein; captured from the Rhone), the upper reaches of the Main, beyond Schweinfurt and the Vosges Mountains, captured from the Meuse, to its watershed. \\
        \end{tabular}
    };
    \label{fig:ex-5ad2990cd7d075001a429b61}
\end{figure*}

\begin{figure*}[ht]
    \center
    \tikz\node[draw=black!40!lightblue,inner sep=1pt,line width=0.3mm,rounded corners=0.1cm]{
        \begin{tabular}{p{.95\textwidth}}
            \textbf{\discability{}}: 1.30 \textbf{\diff{}}: -2.92 \textbf{Feasibility}: 1.00 \textbf{Mean Accuracy}: 0.907                                                                                                                                                                                                                                                                                                                                                                                                                                                                                                                                                                                                                                                                                                                                                                                                                                                                                                                                                                                                                          \\
            \textbf{Validity}: wrong \textbf{Reason}: answer\_set\_incomplete                                                                                                                                                                                                                                                                                                                                                                                                                                                                                                                                                                                                                                                                                                                                                                                                                                                                                                                                                                                                                                                                       \\
            \textbf{Wikipedia Page}: \underline{French\_and\_Indian\_War} \textbf{Question ID}: 5733d13e4776f419006612c5                                                                                                                                                                                                                                                                                                                                                                                                                                                                                                                                                                                                                                                                                                                                                                                                                                                                                                                                                                                                                            \\
            \textbf{Question}: How successful was initial effort by Braddock?                                                                                                                                                                                                                                                                                                                                                                                                                                                                                                                                                                                                                                                                                                                                                                                                                                                                                                                                                                                                                                                                       \\
            \textbf{Official Answer}: disaster; he was defeated in the Battle of the Monongahela \textbf{|} disaster \textbf{|} was a disaster \textbf{|} he was defeated \textbf{|} None succeeded                                                                                                                                                                                                                                                                                                                                                                                                                                                                                                                                                                                                                                                                                                                                                                                                                                                                                                                                                 \\
            \textbf{Context}: In 1755, six colonial governors in North America met with General Edward Braddock, the newly arrived British Army commander, and planned a four-way attack on the French. None succeeded and the main effort by Braddock was a disaster; he was defeated in the Battle of the Monongahela on July 9, 1755 and died a few days later. British operations in 1755, 1756 and 1757 in the frontier areas of Pennsylvania and New York all failed, due to a combination of poor management, internal divisions, and effective Canadian scouts, French regular forces, and Indian warrior allies. In 1755, the British captured Fort Beausejour on the border separating Nova Scotia from Acadia; soon afterward they ordered the expulsion of the Acadians. Orders for the deportation were given by William Shirley, Commander-in-Chief, North America, without direction from Great Britain. The Acadians, both those captured in arms and those who had sworn the loyalty oath to His Britannic Majesty, were expelled. Native Americans were likewise driven off their land to make way for settlers from New England. \\
        \end{tabular}
    };
    \label{fig:ex-5733d13e4776f419006612c5}
\end{figure*}

\begin{figure*}[ht]
    \center
    \tikz\node[draw=black!40!lightblue,inner sep=1pt,line width=0.3mm,rounded corners=0.1cm]{
        \begin{tabular}{p{.95\textwidth}}
            \textbf{\discability{}}: 1.79 \textbf{\diff{}}: -2.27 \textbf{Feasibility}: 1.00 \textbf{Mean Accuracy}: 0.882                                                                                                                                                                                                                                                                                                                                                                                                                                                                                                                                                                                          \\
            \textbf{Validity}: correct \textbf{Reason}: high\_probability + no\_answer                                                                                                                                                                                                                                                                                                                                                                                                                                                                                                                                                                                                                              \\
            \textbf{Wikipedia Page}: \underline{Imperialism} \textbf{Question ID}: 5730a314069b5314008321ed                                                                                                                                                                                                                                                                                                                                                                                                                                                                                                                                                                                                         \\
            \textbf{Question}: What did the the Europeans think the peoples in the tropics were in need of?                                                                                                                                                                                                                                                                                                                                                                                                                                                                                                                                                                                                         \\
            \textbf{Official Answer}: guidance \textbf{|} guidance and intervention \textbf{|} guidance and intervention \textbf{|} guidance and intervention \textbf{|} guidance                                                                                                                                                                                                                                                                                                                                                                                                                                                                                                                                   \\
            \textbf{Context}: According to geographic scholars under colonizing empires, the world could be split into climatic zones. These scholars believed that Northern Europe and the Mid-Atlantic temperate climate produced a hard-working, moral, and upstanding human being. Alternatively, tropical climates yielded lazy attitudes, sexual promiscuity, exotic culture, and moral degeneracy. The people of these climates were believed to be in need of guidance and intervention from the European empire to aid in the governing of a more evolved social structure; they were seen as incapable of such a feat. Similarly, orientalism is a view of a people based on their geographical location. \\
        \end{tabular}
    };
    \label{fig:ex-5730a314069b5314008321ed}
\end{figure*}

\begin{figure*}[ht]
    \center
    \tikz\node[draw=black!40!lightblue,inner sep=1pt,line width=0.3mm,rounded corners=0.1cm]{
        \begin{tabular}{p{.95\textwidth}}
            \textbf{\discability{}}: 1.65 \textbf{\diff{}}: -2.70 \textbf{Feasibility}: 1.00 \textbf{Mean Accuracy}: 0.913                                                                                                                                                                                                                                                                                                                                                                                                                                                                           \\
            \textbf{Validity}: correct \textbf{Reason}: high\_probability + no\_answer                                                                                                                                                                                                                                                                                                                                                                                                                                                                                                               \\
            \textbf{Wikipedia Page}: \underline{Ctenophora} \textbf{Question ID}: 5725cb33271a42140099d1dc                                                                                                                                                                                                                                                                                                                                                                                                                                                                                           \\
            \textbf{Question}: What makes ctenophores different from all other animals?                                                                                                                                                                                                                                                                                                                                                                                                                                                                                                              \\
            \textbf{Official Answer}: by having colloblasts \textbf{|} having colloblasts \textbf{|} colloblasts                                                                                                                                                                                                                                                                                                                                                                                                                                                                                     \\
            \textbf{Context}: Ctenophores form an animal phylum that is more complex than sponges, about as complex as cnidarians (jellyfish, sea anemones, etc.), and less complex than bilaterians (which include almost all other animals). Unlike sponges, both ctenophores and cnidarians have: cells bound by inter-cell connections and carpet-like basement membranes; muscles; nervous systems; and some have sensory organs. Ctenophores are distinguished from all other animals by having colloblasts, which are sticky and adhere to prey, although a few ctenophore species lack them. \\
        \end{tabular}
    };
    \label{fig:ex-5725cb33271a42140099d1dc}
\end{figure*}

\clearpage