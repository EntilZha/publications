\section{Related Work}
\label{ch:isicle:rel}

\name{} draws together two primary threads: we use \irt{} to
understand datasets, which has been applied to other \abr{nlp} tasks,
and apply it to improving leaderboards.
%
%In this section and the next we review alternative applications of \irt{} in \abr{nlp} and alternative approaches for improving leaderboards.
%
Finally, we explore how the insights of \irt{} can improve not just
the analysis of test sets but to improve the \emph{construction} of
test sets.

\paragraph{\textbf{\abr{irt} in \abr{nlp}}}
\irt{} is gaining traction in machine learning
research~\citep{martinez2016ml,martinez2019irt} where automated metrics can be
misleading~\citep{sedoc2019chateval}: machine
translation~\citep{hopkins2013competitions} and chatbot
evaluation~\citep{sedoc2020irt}.
%
Concurrent with our work, \citet{vania2021compare} compare \nlp{} test sets with \irt{}.
Closest to our work in \nlp{} is \citet{otani2016aggregation}, who
rank machine translation \subjs{} and compute correlations with gold
scores.  Similarly, \citet{martinez2020indicators} use \irt{} on
non-language \abr{ai} video game benchmarks.
%
Just as we use \irt{} to identify difficult or easy \itms{},
\citet{lalor2016irt} create challenge sets for textual entailment.
%
We
test \irt{} as a way to guide annotation, but it can also
train \nlp{} models; for example, deep models learn ``easy'' examples
faster~\citep{lalor2018diff} and maintain test accuracy when training
data are down-sampled~\citep{lalor2019latent}.

\paragraph{\textbf{Improving Leaderboards}}
%
The rise \nlp{} leaderboards has encouraged critical thought into
improving them~\citep{linzen2020progress}, improving evaluation more
broadly~\citep{eger2020workshop}, and thoughtful consideration of
their influence on the direction of
research~\citep{sculley2018curse,dotan2020value}.
%
\name{} aims make leaderboard
yardsticks~\citep{hernandez2020yardsticks} more reliable,
interpretable, and part of curating the benchmark itself.
%
In line
with our reliability goal, just as statistical tests should appear in
publications~\citep{dror2018guide,Dodge2019ShowYW}, they should be
``freebies'' for leaderboard
participants~\citep{ethayarajh2020utility}.  Alternatively,
\citet{hou2019leader} posit that leaderboards could be automatically
extracted from publications.
%
How to aggregate multi-task
benchmarks~\citep{wang2018glue,wang2019superglue,fisch2019mrqa} and multi-metric benchmarks~\citep{ma2021dynaboard} is an
open question which---although we do not address---is one use for
\irt{}.


This work implicitly argues that leaderboards should be
continually updated.  As a (static) leaderboard ages, the task(s)
overfit~\citep{recht2019generalize} which---although
mitigable~\citep{blum2015ladder,andersonCook2019host}---is best solved
by continually collecting new data~\citep{kiela2021dynabench}.
%
Ideally, new data should challenge models through adversarial
collection~\citep{wallace2018trick,nie2019adversarial} and related
methods~\citep{Gardner2020-gn}.  However, if making an easy
leaderboard more difficult is possible, the
leaderboard has outlived its helpfulness and should be retired~\citep{voorhees1999trec8}.

Part of our work centers on alternate task efficacy rankings, but this
na\"ively assumes that task efficacy is the sole use case of
leaderboards.
%
Indeed, focusing solely these factors can mislead the
public~\citep{paullada2020data} and may not reflect human language
capabilities~\citep{schlangen2020targeting}.
%
Leaderboards are also well positioned to provide incentive structures
for participants to prioritize fairness~\citep{bender2018data} and
efficiency~\citep{strubell2019energy,schwartz2020green,min2021efficientqa}
or incorporate testing of specific
capabilities~\citep{ribeiro2020checklist,Dunietz2020-ty}.
%
To enable these more nuanced analyses, leaderboards should accept
runnable models rather than static
predictions~\citep{ma2021dynaboard}.

\paragraph{\textbf{Active Learning}}
Beyond \irt{}, the analysis of training dynamics and active learning~\citep{settles09active} is helpful for actively sampling specific \itms{} or identifying low-quality \itms{}~\citep{brodley1999mislabel}.
For example, \citet{swayamdipta2020cartography} and \citet{pleiss2020aum} propose alternative training dynamics-based methods for identifying difficult \itms{} as well annotation errors.
Even closer to goals, \citet{rahman2020active} use active learning to build a test collection.
%
Explicitly measuring how effectively examples separate the best \subj{} from the rest allows test set curators to ``focus on the bubble''~\citep{boydgraber2020nerds}, prioritizing examples most likely to reveal interesting distinctions between submitted systems.
%
%Generally, the family of uncertainty sampling methods~\citep{lewis1994uncertainty} are similar to maximizing item information (\S\ref{ch:isicle:sampling}), but our primary application is towards collecting evaluation data rather than training \nlp{} models.

\paragraph{\textbf{Alternate Formulations}}
%
\irt{} is an example of convergent evolution of models that
predict \subj{} action given an \itm{}.
%
Ideal point models~\cite{poole2017voting} consider how a legislator (\subj{})
will vote on a bill (\itm{}) and use a similar mathematical formulation.
%
The venerable \abr{elo} model~\cite{glickman-99} and modern
extensions~\cite{herbrich-07} predict whether a player (\subj{}) will
defeat an opponent (\itm{}) with, again, a similar mathematical model.
%
Certain \irt{} models can also be formulated as nonlinear mixed
models~\cite{rijmen2003nonlinear}, where the \itm{} parameters are fixed effects
and the latent \subj{} parameters are random effects.
%
This allows for comparisons between \irt{} models and other mixed effects models
under a consistent framework.
    %
    {\bf \pl{1}} and {\bf \pl{2}} can be formulated as nonlinear mixed models, and {\bf \pl{3}} can be formulated as a discrete mixture model over~\itm{}s.
%
As we discuss further in the next section, \name{}'s application of
\irt{} can further be improved by adopting interpretable extensions of
these models.


% Unused
%\citep{urbano2018simulation,urbano2019testing}
%old cite for testing~\citep{hull1993test}
%older sig test for algorithm comparison~\citep{bouckaert2004sig}
%rasch goodness of fit (maybe check for model eval)~\citep{andersen1973fit}
%EDM paper that has nice experiments/evaluation methods, different link functions, and links to some human data~\citep{wu2020virt}
%Generalization across multiple bert runs with similar test set perf~\citep{mccoy2020feather}
%Varied reasons for testing~\citep{moreno1984cat}
