\section{Modeling Experiments}
\label{sec:exp}

\charm{} improves over a \bert{} model in most tasks.

\begin{table*}[t]
    \small
    \centering
    \IfFileExists{2020_emnlp_curiosity/commit_auto_fig/experiment-table.tex}{\begin{tabular}{l r r r r r r r r r r r r}
    \toprule                 & \multicolumn{2}{c}{Fact Rank (\textsc{mrr})} & \multicolumn{2}{c}{Utt. Act (\fone{})} & \multicolumn{2}{c}{Policy Act (\fone{})} & \multicolumn{2}{c}{Like (Accuracy)}                                                   \\
    \cmidrule(lr){2-3}\cmidrule(lr){4-5}\cmidrule(lr){6-7}\cmidrule(lr){8-9}
    Model                    & Val                                          & Test                                   & Val                                      & Test                                & Val   & Test           & Val   & Test           \\
    \midrule  Majority Class & \abr{n/a}                                    & \abr{n/a}                              & 0.602                                    & 0.604                               & 0.491 & 0.494          & 0.690 & 0.681          \\
    \abr{e2e} \bert{}        & 0.420                                        & 0.418                                  & 0.794                                    & 0.795                               & 0.635 & 0.631          & 0.829 & \textbf{0.822} \\
    \charm{}                 & 0.546                                        & \textbf{0.546}                         & 0.845                                    & \textbf{0.847}                      & 0.682 & \textbf{0.682} & 0.826 & 0.815          \\
    $-$ context              & 0.516                                        & 0.506                                  & 0.838                                    & 0.842                               & 0.664 & 0.664          & 0.824 & 0.820          \\
    \bottomrule\end{tabular}}{\begin{tabular}{l r r r r r r r r r r r r}
    \toprule                 & \multicolumn{2}{c}{Fact Rank (\textsc{mrr})} & \multicolumn{2}{c}{Utt. Act (\fone{})} & \multicolumn{2}{c}{Policy Act (\fone{})} & \multicolumn{2}{c}{Like (Accuracy)}                                                   \\
    \cmidrule(lr){2-3}\cmidrule(lr){4-5}\cmidrule(lr){6-7}\cmidrule(lr){8-9}
    Model                    & Val                                          & Test                                   & Val                                      & Test                                & Val   & Test           & Val   & Test           \\
    \midrule  Majority Class & \abr{n/a}                                    & \abr{n/a}                              & 0.602                                    & 0.604                               & 0.491 & 0.494          & 0.690 & 0.681          \\
    \abr{e2e} \bert{}        & 0.420                                        & 0.418                                  & 0.794                                    & 0.795                               & 0.635 & 0.631          & 0.829 & \textbf{0.822} \\
    \charm{}                 & 0.546                                        & \textbf{0.546}                         & 0.845                                    & \textbf{0.847}                      & 0.682 & \textbf{0.682} & 0.826 & 0.815          \\
    $-$ context              & 0.516                                        & 0.506                                  & 0.838                                    & 0.842                               & 0.664 & 0.664          & 0.824 & 0.820          \\
    \bottomrule\end{tabular}}
    \caption{
        The \charm{} model outperforms end-to-end \bert{} on most tasks.
        We compare fact selection with \mrr{}, dialog act prediction with micro-averaged \fone{}, and like prediction with accuracy.
        Ablating dialog history degrades context-dependent tasks (fact selection and policy act prediction), but not tasks more dependent on one message.
    }
    \label{tab:experiments}
\end{table*}

\subsection{Evaluation}
We evaluate each sub-task with separate metrics.
Fact selection is evaluated with mean reciprocal rank (\mrr{}).
For utterances with at least one selected fact, we compute the \mrr{} using the selected facts as relevant documents.
We compare like prediction with binary classification accuracy.
For utterance and policy act prediction, we compare models with micro-averaged \fone{} scores so that frequent classes are weighted more heavily.
For each metric, we report validation and test set scores.

\subsection{Baselines}
\bert{}~\citep{Devlin2018BERTPO} is a standard baseline for many \abr{nlp} tasks.
We use a multi-task extension of an uncased \bert{} model as our primary baseline and fine-tune it for our unique set of tasks~(\mtbert).
Specifically, we use the \abr{cls} representation of each utterance to replace the \hre{} representation as a time-distributed input to the same multi-task decoders~(Section \ref{subsec:method:task}).
The context-less \charm{} ablation replaces the dialog contextualizer \abr{lstm} with a per-timestep projection layer.
Lastly, we report majority class accuracy for classification tasks.

\subsection{Discussion}

The proposed \charm{} model for conversational curiosity is more effective than \mtbert{} for most of the tasks in \rover{} (Table~\ref{tab:experiments}).
Specifically, \charm{} improves significantly in fact prediction ($13$ \mrr{} points) and both dialog act prediction tasks ($5$ \fone{} points), demonstrating the efficacy of the structural encoding of the various input modalities.
Generally, models accurately predict utterance acts and likes, but their \mrr{} and \fone{} scores on fact selection and policy act prediction is comparatively worse.
To a degree, this is expected since there is not always one best fact or one best action to take as the assistant; there may be various reasonable choices, which is common in information retrieval tasks.
Nonetheless, models that specifically reason about the relationship between prior knowledge and entities would likely yield improvement.
For example, \citet{Liu2018KnowledgeDF} predict the most relevant unmentioned entity while~\citet{Lian2019LearningTS} model a posterior distribution over knowledge.
We leave these improvements to future work.
